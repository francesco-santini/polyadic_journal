\documentclass[submission,copyright,creativecommons]{eptcs}
\providecommand{\event}{MeMo 2015} % Name of the event you are submitting to
\usepackage{breakurl}             % Not needed if you use pdflatex only.

\usepackage{latexsym,enumerate}
\usepackage{amsmath, amssymb, amsxtra, amsfonts}
\usepackage{graphicx}
\usepackage{url}
\usepackage{xspace}
\usepackage{booktabs}
\usepackage{hyperref}
\usepackage{marvosym}
\usepackage{pxfonts}
\usepackage{framed}
%\usepackage{times}
%% New package
\usepackage{subfiles}
\usepackage{color}

\def\odiv{\, {\ominus\hspace{-6.8pt} \div} \,}
\newcommand{\SCCP}{\texttt{SCCP}\xspace}
\newcommand{\CCP}{\texttt{CCP}\xspace}
\newcommand{\CCS}{\texttt{CCS}\xspace}

\newcommand\remove[1]{}
%to remove the proofs on exists and \otimes completeness
\newcommand{\Exists}{\exists\!\!\exists}
%\newcommand{\rrarrow}{\longrightarrow}
\newcommand{\la}{\langle}
\newcommand{\ra}{\rangle}
\newcommand{\rarrow}{\rightarrow}
\newcommand{\longar}{\longrightarrow}
\newcommand\deff{\mathit{def}}
\newcommand\conn{\mathit{con}}
\newcommand\mahh{\mathit{mah}}
\newcommand{\lan}{\langle}
\newcommand{\ran}{\rangle}
\newcommand{\os}{[\![}
\newcommand{\cs}{]\!]}
\newcommand{\rrarrow}{\longrightarrow}
\newcommand{\hiding}{\\exists}
\newcommand{\restr}{\mid}
\newcommand{\lrarrow}{{\longrightarrow}}
\newcommand{\tell}{{\bf tell}}
\newcommand{\atell}{{\bf atell}}
\newcommand{\ask}{{\bf ask}}
\newcommand{\ostop}{{\bf stop}}
\newcommand{\nask}{{\itshape nask}}
\newcommand{\update}{{\bf update}}
\newcommand{\sat}{{\itshape sat}}
\newcommand{\nil}{{\itshape nil}}
\newcommand{\true}{{\itshape true}}
\newcommand{\false}{{\itshape false}}
\newcommand{\vars}{{\itshape Var}}
\newcommand{\form}{P \; sat \; \phi}
\newcommand{\timeout}{{\bf timeout}}
\newcommand{\NI}{\noindent}
\newcommand{\bis}{\, \dot{\approx} \,}
%\newenvironment{program}{\tt \begin{tabbing}pro\={\tt pro}\= clause \kill}{\end{tabbing}}
\def\ent{\vdash}
\def\ent{\vdash}
\def\0{{\mathbf 0}}
\def\1{{\mathbf 1}}
\def\C{{\mathcal C}}

%\addtolength{\floatsep}{-4mm}
%\addtolength{\textfloatsep}{-4mm}

\makeatletter
\renewcommand{\paragraph}{%
	\@startsection{paragraph}{4}%
	{\z@}{1ex \@plus 1ex \@minus .2ex}{-1em}%
	{\normalfont\normalsize\bfseries}%
}
\makeatother


%% Commands by Luis
\usepackage{commands}


%%  Tell me if you like this title. It's ok if you want to keep the old one.
%%
\title{Recovering a Labelled Semantics\\ for Soft CCP with Local Variables\thanks{Research 
		partially supported 
		%by the EU FP7-ICT IP 257414 ASCENS, 
		by the MIUR PRIN 2010LHT4KM CINA
		and 2010XSEMLC ``Security Horizons''.
	}
}



\author{Fabio Gadducci
	\institute{Dipartimento di Informatica, \\ Universit\`a di Pisa, Italy}
	\email{gadducci@di.unipi.it}
	\and
	Francesco Santini
	\institute{IIT-CNR, \\
		Pisa, Italy}
	\email{\quad francesco.santini@iit.cnr.it}
}
\def\titlerunning{Recovering a Labelled Semantics for Soft CCP}
\def\authorrunning{F. Gadducci \& F. Santini}


\begin{document}



%\author{Fabio Gadducci\inst{1} and Francesco Santini\inst{2}}
%\institute{Dipartimento di Informatica, Universit\`a di Pisa\\
%\email{gadducci@di.unipi.it}
%\and Istituto di Informatica e Telematica, CNR Pisa\\
%\email{francesco.santini@iit.cnr.it}
%}

\maketitle


\paragraph{Extended Abstract.}
The methodological advance of the seminal article on \emph{relative pushouts}~\cite{Leifer:00:CONCUR} is the recognition that, 
while the dynamics of a computational system is nowadays more often specified operationally by means of a reduction system (RS), it should be possible to derive a labelled semantics in order to take advantage of the tools for checking observational semantics.

The proposal in~\cite{Leifer:00:CONCUR} identified a notion of \emph{minimal} label for building from a RS a labelled transition system (LTS). Unfortunately, checking such minimality proved to be a difficult task, also for most basic calculi such as CCS~\cite{BonchiGK09, Milner06}.
%
 One of the recent advances~\cite{BonchiGM14} suggests to remove the requirement of minimality, in order to characterize a class of LTSs verifying weaker coherence requirements, yet ensuring the adequacy of the observational equivalence.

One of the testbed has been \emph{Concurrent Constraint Programming} (CCP)~\cite{popl91},  a language based on a shared-memory communication pattern: processes interact by either posting or checking partial information, represented as constraints in a global store. 
%
CCP belongs to the larger family of process calculi, thus a syntax-driven operational semantics 
represents the computational steps. For example, the term $\tellp{c}$ is the process that posts $c$ in the store, 
and the term $\askp{c}{P}$ is the process that executes $P$ if $c$ can be derived from the information 
in the store. 
%Furthermore $P \parallel Q$ represents the parallel execution of $P$ and $Q$, and $\localp{x}{P}{}$ 
%denotes a process $P$ with a private/local variable $x$.

The formalism is parametric with respect to the entailment relation. Under the name of \emph{constraint system}, 
the information recorded on the store is structured as a partial order (actually, a lattice) $\leq$, where $c\leq d$ means that $c$ can be derived from $d$.
%
Under a few requirements over such systems, CCP has been provided with (coincident) operational and denotational semantics. 



A key aspect of CCP is the \emph{idempotency} of constraint composition:
%(i.e., the lub $\sqcup$, given by the closure of the union~\cite{popl91}): 
adding an information twice does not change the store. 
%
In the soft variant of the formalism (Soft CCP, SCCP~\cite{scc}),
constraint systems may distinguish the number of occurrences of a  
piece of information.
% i.e., we may have $c \sqcup c \not= c$. 
% and the lattice of information is replaced by an ordered monoid, there the monoidal operation represents information combination.
%
Dropping idempotency requires a full reworking of the theory. Although an operational semantics for SCCP has been devised \cite{scc},
neither the denotational nor the labelled one has been reintroduced. 


%As a language, SCCP has been used as a specification formalism for agents collaborating 
%via a shared knowledge basis, possibly with temporal features~\cite{coordination08}, with applications to the negotiation of Service Level %Agreements~\cite{ccpi,fun11}, or the enforcement of ACL-based access control~\cite{fun14}.


The work in \cite{popl91} establishes a denotational semantics
for CCP and an equational theory for infinite agents. More recently, in \cite{pippo} the authors
prove that the axiomatisation is underlying a specific weak bisimilarity among agents,
thus providing a clear operational understanding.
%
The key ingredients are a complete lattice as the domain of the store, with least upper bound for constraint
combination, and a notion of compactness such that domain equations for 
the parallel composition of 
recursive agents are well-defined.
% (more on this later).
%
The soft version~\cite{scc} drops the upper bound for combination
in exchange of a monoidal operator. Thus, the domain is just a (not
necessarily complete) partial order, possibly with finite meets and a residuation operator (a kind of
inverse of the monoidal one) in order to account for algorithms concerning constraint propagation.
%
Indeed, the main use of SCCP has been in the generalisation of classical constraint satisfaction
problems, hence 
%the emphasis on e.g. local consistency techniques~\cite[\S~3]{handbook}, and 
the lack of investigation about denotational semantics.




In~\cite{coord} we connected the works on the soft~\cite{scc} and the classical 
(also indicated in the literature as ``crisp'')~\cite{pippo,popl91} 
paradigm by investigating
a labelled (and an unlabelled) semantics for a deterministic fragment of SCCP. In particular, the result 
was a mix of those investigated in the two communities, namely, a monoid whose underlying set of elements
form a complete lattice.  Residuation theory 
provided an elegant way to define a weak inverse operator of tensor $\otimes$ with the purpose to  determine the 
minimal information that enables the firing of actions in the LTS, thus distilling a suitable notion of labelled transition.

The operational semantics chosen in~\cite{coord} favoured a global view of  variables, namely, the agent $\exists_x A$, roughly 
corresponding to an existential quantification over the variable $x$ of the agent $A$, would evolve to the agent 
$A[^y/_x]$, for $y$ a fresh variable (with respect to the agent $A$ and the store it is valued in).
In this work we consider instead local variables, where the hiding operator carries some information on the variables it abstracts.
More precisely, according to~\cite{extendedHiding} we consider an extended operator $\exists_x^\sigma$, for $\sigma$ the local store.
Thanks again to the residuation operator, the rule for the extended hidings can be defined as {\bf Hide} in Tab.~\ref{fig:hiding}. The intuition is that variable $x$ may be local to a component $\exists_x \sigma'$ of the store $\sigma$, yet visible at a global level: we must then evaluate $A$ in the store
when the local $x$ is hidden, yet the possible duplications are removed (e.g., $\exists_x \sigma'$ may already occur in the global store $\sigma$). The final store intuitively contains in $\sigma_1$ the original $\sigma'$ increased by what has been added by the step. 

In the labelled version of the rule ({\bf L-Hide} in Tab.~\ref{fig:hiding}) we rename the global $x$ with a fresh variable $z$, instead of hiding $x$ in the global store, as we do in the corresponding unlabelled rule. We accomplish this in order
to keep track of the global $x$ in $\alpha$: finally, in the result we restore the occurrences of $z$ back to $x$.
%, i.e., $\alpha[x/z]$.

The work is rounded up by a preliminary correspondence results between fair computations and bisimulation for soft CCP with 
local variables.

%In this setting CCP has much to offer, besides to other applications as the negotiation of Service Level Agreements~\cite{ccpi,fun11}, or the enforcement of ACL-based access control~\cite{sefm12}.
%

\vspace{-.5cm}
\def\odiv{\; {\ominus\hspace{-6pt} \div} \;}
\def\odivvv{\; {\ominus\hspace{-6pt} \div} \;}
\begin{table}
\begin{center}
	\scalebox{0.9}{
		\begin{tabular}{llll}
			$\frac {\displaystyle \langle A, \sigma' \otimes \exists_x \sigma_0
				\rangle \rrarrow \langle B, \sigma'' \rangle \ \ \ \mbox{  \textit{with}  } \sigma_0 = \sigma \odiv \exists_x \sigma'} 
			{\displaystyle\langle \exists^{\sigma'}_x A,
				\sigma \rangle \rrarrow \langle \exists^{\sigma_1}_x B, \sigma_0 \otimes \exists_x \sigma_1\rangle \ \ \ \mbox{  \textit{with}  } \sigma_1 = \sigma'' \odiv \exists_x \sigma_0}$
			&\bf{Hide} &\vspace{0.2cm} \\ 
 			%$\frac {\displaystyle \la A[z/x], c[z/x] \otimes \sigma \ra \xrightarrow{\;  \alpha \;} \la
			%B,  c' \otimes \sigma \otimes \alpha \ra} {\displaystyle\la \exists^c_x A,
			%\sigma\ra \xrightarrow{\;  \alpha \;} \la \exists^{c'[x/z]}_x B[x/z], \exists_x(c'[x/z]) \otimes \sigma \otimes \alpha \ra} \ \
			% {\it x\not\in fv(c'), z \not\in (fv(A) \cup fv(c \otimes \sigma \otimes \alpha)) }$
			$\frac {\displaystyle \la A, \sigma' \otimes \sigma_0[^z/_x] \ra \xrightarrow{\;  \alpha \;}
				\la B,  \sigma" \ra\ \ \ \mbox{  \textit{with}  } \sigma_0 = \sigma \odiv \exists_x \sigma', 
				                     x\not\in sv(\alpha), z \not\in fv(A) \cup sv(\sigma)\cup sv(\sigma')}
			{\displaystyle\la \exists^{\sigma'}_x A, \sigma\ra \xrightarrow{\;  \alpha[^x/_z] \;}
				\la \exists^{\sigma_1}_x B, \sigma_0 \otimes \alpha[^x/_z] \otimes \exists_x\sigma_1 \ra
				\ \ \ \mbox{  \textit{with}  } \sigma_1 = \sigma" \odiv (\alpha \otimes \sigma_0[^z/_x])}$
			&\bf{L-Hide}&
		\end{tabular}
	}
\end{center}
%\hfill
\caption{Unlabelled and labelled version of the rule for the local hiding of variables.}
\label{fig:hiding}
\end{table}


%We shall argue that one of the advantages of the new technique over the previous approach is that it may reduce 
%the number constraint entailment checks that are needed to decide the equivalence. 

%
%We comply to our design criteria with the fact the processes in the extended language can be seen as extensive, monotonic functions as well as formulae of a persistent version of affine-linear logic.  We first introduce a notion of constraint systems with emphasis on multiplication of information. We then introduce a new complete and sound co-inductive reasoning technique to prove observational process equivalence. We show intuitive CCP equalities do not hold any longer in CCP.  We show that a family of soft-constraint languages can be seen as instances of the calculus here presented and hence they can benefit from the co-inductive techniques here proposed.  

%For this reason we will refer to $ccp$ with multiplicity as soft $ccp$ (sccp).



%
%The advent of phenomena such as social networks have have changed substantially concurrent and distributed systems
%
% In past research on concurrent distributed systems the emphasis has mostly been on consistency, fault tolerance, resource management and related topics; these aspects were all characterized by interaction between processes. The new era of concurrent systems is marked by the importance of managing access to information to a much greater degree than before. 


%% Argue CCP is the ideal calculus to model  posting in shared medium.

%%The argue than posting multiplicity (non idempotency) is important


%% Argue that multiplicity allows for wider phenomena without losing the declaretive nature
%% of CCP:  Programs are formulae of a persistent version of (affine) linear logic. Also
%% Confluence still holds.

%%  Contribution: A co-inductive theory of the calculus. Technical issues: Subtle issue of copying stores.
%%  

%%




%
%The title, we agree, is a bit pretentious$\ldots$ It seems to wipe out, in a second, various years of work
%on the soft paradigm!
%%
%Yet, it is an homage to the seminal paper by Prakash and others that still bears some relevance
%for the current research.
%%
%Let us try to be more precise. The work in \cite{popl91} was establishing a denotational semantics
%for~\textsc{ccp} and a equational theory for infinite agents. More recently, \cite{pippo} was able
%to show that the equational axioms were underlying a specific weak bisimilarity among agents,
%thus providing a clear operational understanding.
%%
%The key ingredients were a complete lattice as the domain of the store, with least upper bound for constraint
%combination, and a notion of compactness such that domain equations for the parallel composition
%of recursive agents would be well-defined (more on this later).
%%
%On the contrary, the soft version introduced in~\cite{scc} dropped the upper bound for combination
%in exchange of a more general monoidal operator. Thus, the domain was potentially just a (not
%necessarily complete) partial order, possibly with finite meets and a residuation operator (kind of
%inverse of the monoidal one) in order to account for algorithms concerning constraint propagation.
%%
%Indeed, the main use of soft constraint has been as a generalisation of classical constraint satisfaction
%problems, hence the emphasis on e.g. local consistency techniques~\cite[\S~3]{handbook}, and the
%lack of investigation about e.g compactness and denotational semantics.
%
%The aim of the paper is thus to connect the work on the soft and crisp paradigm by investigating both
%a denotational and an operational semantics for soft CCP with recursion. The do aim will be a mix of
%those investigated in the two communities, namely, an infinitary monoid whose underlying set of elements
%form a compete lattice. We will hone recast first of all the notion of compactness, and then the semantics,
%for soft CCP, thus making the work a direct extension of the proposal for the crisp language.

\bibliographystyle{eptcs}
\bibliography{softccp}

%\newpage
%\subfile{appendix.tex}

\end{document}
