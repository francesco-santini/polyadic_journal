\documentclass[main.tex]{subfiles}
\begin{document}

\section{An Axiomatisation of Simple Agents for Weak Bisimilarity}\label{sec:axiomatiastion}
Once the behaviour of an agent is captured by an observational equivalence, 
it is natural to look for laws characterising it. 
Given its correspondence with the standard equivalence via fair computations,
weak bisimilarity is the preferred behavioural semantics for soft CCP. 
%
%in this section we shall discuss some axioms for weak bisimilarity in soft CCP. 
A sound and complete axiomatisation was proposed for CCP in~\cite{popl91}. 
%
However, the lack of idempotence in the soft formalism makes unsound some of the axioms  
there, since posting a constraint twice is different from adding it just
once. 
As an example, law \emph{L1} of~\cite{popl91} states that
$\ask(c) \rightarrow \tell(d) = \ask(c) \rightarrow (\tell(c) \parallel \tell(d))$, 
which is clearly false in the soft formalism.

Nevertheless, most of the axioms in~\cite{popl91} can be recovered, as long as we mix together
the various operators available in a CLIM.\footnote{Our axioms follow closely those of~\cite[p.341]{popl91},
replacing  the law \emph{L1} mentioned above with $\tell(\bot) = \ostop$ and dropping altogether the laws 
\emph{L11} and \emph{L12}.}
%
So, let an agent be simple if it is finite and contains no occurrence of an existential quantifier.
Our set of axioms for simple agents of SCCP in presented in Figure \ref{fig:axioms2}. 

As for sequential processes, in Eqs.~\ref{eq:stop1}-\ref{eq:stop2} we present the axioms involving the $\0$
and in Eqs.~\ref{eq:ask1}-\ref{eq:ask2} those related to \emph{ask}.
Those concerning the parallel composition operator are instead represented in Eqs.~\ref{eq:parall1}-\ref{eq:parall3}. 
Distributivity of \emph{tell} and \emph{ask} are presented in Eqs.~\ref{eq:tell1}-\ref{eq:dis1},
while Eqs.~\ref{eq:dis2}-\ref{eq:dis3}  show how further simplify the combination of
\emph{tell} and \emph{ask} through parallel composition. 
 %
%\marginpar{Soundness of these 2 laws must be proved}

%\begin{figure}[t]
%\begin{framed}
%\setcounter{equation}{0}
%\begingroup
%\everymath{\scriptstyle}
%\scriptsize
%%\vspace{-0.35cm}
%
%\begin{minipage}{0.52\linewidth}
%\begin{equation}\label{eq:stop1}
%%\underline{\ask(c) \rrarrow \tell(d) = \ask(c) \rrarrow (\tell(c) \parallel \tell(d))}
%\tell(\bot) = \ostop
%\end{equation}
%\end{minipage}
%\hspace{0.7cm}
%\begin{minipage}{0.34\linewidth}
%\begin{equation} \label{eq:stop2}
%\ask(c) \rarrow \ostop= \ostop
%\end{equation}
%\end{minipage}
%%\vspace{0.13cm}
%
%\begin{minipage}{0.52\linewidth}
%\begin{equation}\label{eq:ask1}
%%\underline{\ask(c) \rrarrow \tell(d) = \ask(c) \rrarrow (\tell(c) \parallel \tell(d))}
%\ask(c) \rarrow \ask(d) \rarrow A= \ask(c \vee d) \rarrow A
%\end{equation}
%\end{minipage}
%\hspace{0.7cm}
%\begin{minipage}{0.34\linewidth}
%\begin{equation}\label{eq:ask2}
%%\ask(c) \rarrow (\ask(d) \rarrow A) = \ask(c \otimes d) \rarrow A
%\ask(\bot) \rarrow A = A
%\end{equation}
%\end{minipage}
%%\vspace{0.13cm}
%
%\begin{minipage}{0.52\linewidth}
%\begin{equation}\label{eq:parall1}
%A \parallel B = B \parallel A
%\end{equation}
%\end{minipage}
%\hspace{0.7cm}
%\begin{minipage}{0.34\linewidth}
%\begin{equation}
%A \parallel (B \parallel C) = (A \parallel B) \parallel C
%\end{equation}
%\end{minipage}
%%\vspace{-0.27cm}
%
%\begin{minipage}{0.52\linewidth}
%\begin{equation}\label{eq:parall3}
%A \parallel \ostop = A
%\end{equation}
%\end{minipage}
%\hspace{0.7cm}
%\begin{minipage}{0.34\linewidth}
%\begin{equation}\label{eq:tell1}
%\tell(c) \parallel \tell(d) =  \tell(c \otimes d)
%\end{equation}
%\end{minipage}
%%\vspace{0.11cm}
%
%\begin{minipage}{\linewidth}
%\begin{equation}\label{eq:dis1}
%\ask(c) \rarrow (A \parallel B) = (\ask(c) \rarrow A) \parallel (\ask(c) \rarrow B)
%\end{equation}
%\end{minipage}
%%\vspace{0.05cm}
%
%\begin{minipage}{\linewidth}
%\begin{equation}\label{eq:dis2}
%%\ask(a) \rarrow \tell(b) \parallel \ask(c) \rarrow \tell(d) = \ask(a) \rarrow \tell(b\otimes d) \mbox{ if $a < c \leq a \otimes b$}
%\ask(a) \rarrow \tell(b) \parallel \ask(a \otimes b) \rarrow \tell(d) = \ask(a) \rarrow \tell(b\otimes d)
%\end{equation}
%\end{minipage}
%
%\begin{minipage}{\linewidth}
%\begin{equation}\label{eq:dis3}
%\ask(a) \rarrow \tell(b) \parallel \ask(c) \rarrow \tell(d) = \ask(a) \rarrow \tell(b) \parallel \ask(c \vee (a \otimes b)) \rarrow \tell(d) \mbox{ if $a \leq c$}
%\end{equation}
%\end{minipage}
%\vspace{0.05cm}
%
%%\begin{equation}
%%\underline{(\ask(a) \rarrow \tell(b)) \parallel (\ask(c) \rarrow \tell(d)) = (\ask(a) \rarrow \tell(b)) \ \ \
%%\ \mathit{if} \ a \leq c, d \leq b}
%%\end{equation}
%%\vspace{-0.9cm}
%
%
%%\scriptsize
%%\begin{equation}\label{eq:dis4}
%%(\ask(a) \rarrow \tell(b)) \parallel (\ask(c) \rarrow \tell(d)) = (\ask(a) \longrightarrow \tell(b)) \parallel (\ask(c \otimes b) \rarrow \tell(d)) \ \ \
%%\ \mathit{if} \ a \leq c
%%\end{equation}
%%\vspace{-0.6cm}
%
%
%%%%%%NOT IN POPL91 BUT IT WAS HERE BEFORE....TO ADD OR NOT?
%%\scriptsize
%%\begin{equation}\label{eq:tell1}
%%\tell(\bot) = \ostop
%%end{equation}
%
%%\begin{equation}\label{eq:dis5}
%%\underline{(\ask(a) \rarrow \tell(b)) \parallel (\ask(c) \rarrow \tell(d)) = (\ask(a) \rarrow \tell(b)) \parallel (\ask(c) \rarrow \tell(d \otimes b) \ \ \ \
%%\mathit{if} \ a \leq d}
%%\end{equation}
%%\vspace{-0.4cm}
%
%\endgroup
%%\vspace{-0.1cm}
%
%\setcounter{equation}{8}
%
%\begingroup
%\everymath{\scriptstyle}
%\scriptsize
%%\vspace{-0.35cm}
%
%%\begin{minipage}{0.37\linewidth}
%%\begin{equation}\label{eq:hid1}
%%\exists_x \tell(c)  = \tell(\exists_x c)
%%\end{equation}
%%\end{minipage}
%%\hspace{0.19cm}
%%\begin{minipage}{0.5\linewidth}
%%\begin{equation}\label{eq:hid2}
%%\exists_x (\ask(c) \rightarrow A) = \ask(\forall_x c) \rightarrow \exists_x A
%%\end{equation}
%%\end{minipage}
%%\vspace{-0.1cm}
%
%%\begin{equation}
%%\underline{\exists_x (\parallel_{i \in I} \ask(c_i)\rarrow \tell(d_i)) = \parallel_{i \in I}  \exists_x (\ask(c_i) \rarrow (\tell(d_i)  \parallel_{j \in I, j \not= i} \ask(c_j) \rarrow \tell (d_j))}
%%\end{equation}
%%\vspace{-0.6cm}
%
%%\begin{equation}\label{eq:hid4}
%%\exists_x (\tell(c) \parallel_{i \in I} \ask(c_i) \rarrow \tell(d_i)) = \tell(\exists_x c) \parallel \exists_x (\parallel_{i \in I} \ask(c\Rightarrow_x c_i) \rarrow \tell(d_i))
%%\end{equation}
%\endgroup
%%\vspace{-0.1cm}
%\end{framed}\caption{Axioms for simple agents.}% and for agents with quantifiers (\ref{eq:hid1}-\ref{eq:hid4})
%\label{fig:axioms2}
%\end{figure}

	\setcounter{equation}{0}


\begin{figure}
\scriptsize
	\begin{flalign}
\tell(\bot) = \ostop\label{eq:stop1}\\
\ask(c) \rarrow \ostop= \ostop\label{eq:stop2}\\
\ask(c) \rarrow \ask(d) \rarrow A= \ask(c \vee d) \rarrow A\label{eq:ask1}\\
\ask(\bot) \rarrow A = A\label{eq:ask2}\\
A \parallel B = B \parallel A\label{eq:parall1}\\
A \parallel (B \parallel C) = (A \parallel B) \parallel C\label{eq:parall2}\\
A \parallel \ostop = A\label{eq:parall3}\\
\tell(c) \parallel \tell(d) =  \tell(c \otimes d)\label{eq:tell1}\\
\ask(c) \rarrow (A \parallel B) = (\ask(c) \rarrow A) \parallel (\ask(c) \rarrow B)\label{eq:dis1}\\
\ask(a) \rarrow \tell(b) \parallel \ask(a \otimes b) \rarrow \tell(d) = \ask(a) \rarrow \tell(b\otimes d)\label{eq:dis2}\\
\ask(a) \rarrow \tell(b) \parallel \ask(c) \rarrow \tell(d) = \ask(a) \rarrow \tell(b) \parallel \ask(c \vee (a \otimes b)) \rarrow \tell(d) \mbox{ if $a \leq c$}\label{eq:dis3}
	\end{flalign}
\caption{Axioms for simple agents.}\label{fig:axioms2}
\end{figure}

\setcounter{equation}{8}


\begin{proposition}
The axioms in Figure \ref{fig:axioms2} are sound and complete for simple agents with respect to weak bisimilarity.
\end{proposition}
%\begin{proof}
%AGGIUNGERE SKETCH
%\end{proof}
%\begin{proof}
%SO FAR, THIS IS FALSE. Indeed, e.g. the agents in $(1)$ are not weakly bisimilar. Similarly, $(10)$, $(12)$ and $(15)$ are false. The problem is linked with the lack of idempotency %of $\otimes$. Law $(13)$, $(14)$, and $(16)$ should be checked. Law $(15)$ is a real problem: it has somehow to be reformulated, otherwise, there is no normal form at all.
%\end{proof}

Soundness is easily checked, while completeness is obtained by mimicking the proof schema adopted for the crisp case, 
exploiting  a normal form that is in fact reminiscent of the one in~\cite[Definition 3.2]{popl91}.\footnote{With respect to the properties 
stated in~\cite[p.342]{popl91}, we weakened property 1 and dropped property 4, the latter being linked to axiom L12.}

%\normalsize
%The completeness of the axiomatisation in Figs.~\ref{fig:axioms1} and~\ref{fig:axioms2} would require a normal form on agents. Indeed, a simple one is available 
%for our set of axioms.
%
%\begin{definition}[Normal form]\label{def:nf}
%An agent $A$ is in normal form if $A = \tell(\bot)$, or $A = \parallel_{i \in I} \ask(c_i)\rarrow \tell(d_i)$ and (2) all $c_i$'s are pairwise different, and (3) $c_i \leq c_j$ implies $d_i \leq c_j$.
%\end{definition}
%
%The two more properties required in \cite{popl91}, i.e. ``$c_i \leq d_i$'' and ``$c_i \leq d_j$, implies $d_i \leq d_j$'', hold only for normal forms over idempotent CLIMs.
%Also property (3) is slightly weaker than in the idempotent variant. Finally, note that the normal form is not unique, intuitively depending on the order of application of the law $(11)$.
%
%\begin{lemma}\label{def:nf}
%Any agent $A$ containing no hiding construct $\exists_x B$ can be converted to normal form using Eq.~\ref{eq:ask1}-Eq.~\ref{eq:dis4}.
%\end{lemma}

%\marginpar{A stronger normal form may be necessary}
%
%\begin{theorem}
%Let $A$ and $B$ be finite agents. Then $\langle A, \bot \rangle \approx \langle B, \bot \rangle$ if and only if $A$ and $B$ have the same normal form.
%\end{theorem}
%
%\begin{proof}
%TO DO [Note that the notion of normal form here is tricky]
%\end{proof}

\begin{lemma}
Let $A$ be an agent. Then either $A = \ostop$ or it can be decomposed as $\bigparallel_{i} \ask(c_i) \rarrow \tell(d_i)$ such that 
\begin{itemize}
\item  $d_i \neq \bot$
\item  $c_i \neq c_j$ for all $i \neq j$
\item $c_i < c_j$ implies $c_i \otimes d_i < c_j$
\end{itemize}
\end{lemma}

%\begin{proof}
Axioms 1-8 guarantee that an agent can be decomposed as $\bigparallel_{i} \ask(c_i) \rarrow \tell(d_i)$
such that $d_i \neq \bot$. Axiom 9 then ensures the second condition. Finally, by applying as long as possible first axiom 11 
and then axiom 10, the final condition is also enforced.
%\end{proof}
\end{document}
