\documentclass[main.tex]{subfiles}
\begin{document}

\section{The Algebraic Background}\label{sec:background}

This section recalls the main notions we are going to need later on. First of all, we present some basic facts 
concerning monoids~\cite{CLIM} enriched over complete lattices. These are used to recast the standard 
presentation of the soft constraints paradigm, and to generalise the classical crisp one.
%
% some constructions that are needed later on
%(such as resolution operator). Then, we present soft constraints systems~\cite{jacm} and the deterministic
%fragment of recursive soft concurrent constraint programming(thus extending~\cite{scc}).

\subsection{Lattice-enriched Monoids}\label{sec:lem}
\begin{definition}[Complete lattices]
A partial order (PO) is a pair $\langle A, \leq \rangle$ such that
$A$ is a set of values and $\leq \,\,\subseteq A \times A$ is a reflexive, transitive, and
anti-symmetric  relation.
A complete lattice (CL) is a PO such that any subset of $A$ has a least upper bound (LUB).
\end{definition}

We denote as $\bigvee X$ the necessarily unique LUB of a subset $X \subseteq A$, and explicitly $\bot$ and $\top$ if
we are considering the empty set and the whole $A$, respectively: the former is the bottom and
the latter is the top of the PO.
%
Obviously, CLs also have the greatest lower bound (GLB) for any subset $Y \subseteq A$, denoted as $\bigwedge Y$.
In the following we fix a CL ${\mathbb C} = \langle A, \leq \rangle$.

\begin{definition}[Compact elements]
An element $a \in A$ is compact (or finite) if whenever $a \leq \bigvee Y$ there exists a finite subset
$X \subseteq Y$ such that $a \leq \bigvee X$.
%
%Let $A^C \subseteq A$ be the set of compact elements of ${\mathbb C}$.
%Then ${\mathbb C}$ is algebraic if $\forall c \in A. c = \bigvee \{ d \in A^C \mid d \leq c\}$.
\end{definition}


Note that for complete lattices the definition of compactness given above coincides with the one using
directed subsets. It will be easier to generalise it, though, to compactness with respect to the monoidal operator (see Definition~\ref{def:compactness}).
%
We let $A^C \subseteq A$ denote the set of compact elements of ${\mathbb C}$. 

\begin{example}
Note that $A^C$ might be trivial. Consider e.g. the CL $\langle [0, 1], \geq \rangle$ (the segment of the reals
with the inverse of the usual order), used for probabilistic constraints~\cite{probabilistic}:
%so that 
only the bottom element $1$ is compact. 
%
As we will see, the situation for the soft paradigm is more nuanced.
\end{example}
%\marginpar{is algebraicity needed?}
%

\begin{definition}[Monoids]
A commutative monoid with identity (IM) is a triple
$\langle A, \otimes, \1 \rangle$ such that $\otimes: A \times A \rightarrow A$ is
a commutative and associative function and 
 \begin{description}
 \item[(identity)] $\forall a \in A.\otimes (a, \1) = a$
 \end{description}
\end{definition}

We will often use an infix notation, such as $a \otimes b$ for $a, b \in A$.
%
The monoidal operator can be defined for any multi-set: it is given 
for a family of elements $a_i \in A$ indexed over a finite, non-empty
set $I$, and it is denoted by
$\bigotimes_{i \in I} a_i$.
%
If for an index set $I$ the $a_i$'s are different,
we write $\bigotimes S$ instead of $\bigotimes_{i \in I} a_i$
for the set $S = \{a_i \mid i \in I\}$.
%
Conventionally, we denote $\bigotimes \emptyset = \bot$.

%smallskip
%In the following we fix a IM ${\mathbb M} = \langle A, \otimes, \1 \rangle$.

We now move our attention to our choice for the domain of values.

\begin{definition}[CL-enriched IMs]
A CL-enriched IM (CLIM) is a triple ${\mathbb S} = \langle A, \leq, \otimes \rangle$ such that
 $\langle A, \leq \rangle$ is a CL,  $\langle A, \otimes, \bot \rangle$ is an IM,
 and
 \begin{description}
 \item[(distributivity)] $\forall a \in A. \forall X \subseteq A.\, a \otimes  \bigwedge X = \bigwedge \{a \otimes x \mid x \in X\}$
 \end{description}
\end{definition}

\begin{remark}
The reader who is familiar with the soft constraint literature may have noticed that we have 
basically rewritten the standard presentation using a CLIM instead of an absorptive 
semiring, recently popularised as c-semiring~\cite{jacm}, where the $a \oplus b$ operator is 
replaced by the binary LUB $a \vee b$. 
%
Besides what we consider a streamlined presentation, the main advantage in the use of 
CLIMs is the easiness in defining the LUB of infinite sets and, as a consequence, the notion of 
$\otimes$-compactness given below. 
%
An alternative solution using infinite sums can be found in~\cite[Section~3]{golan}, 
and a possible use is sketched in \cite{ecai06}.
\end{remark}


%\marginpar{distributivity wrt. $\vee$ or wrt. $\wedge$ coincide?}
%Thanks to distributivity, we can show that
%$\bigotimes$ is monotone, i.e., $\forall j \in I. a_j \leq b_j \implies
%\bigotimes_{i \in I} a_i \leq \bigotimes_{i \in I} b_i$.
%%
%And since $\bot$ is the identity of the monoid,
%monotonicity implies that the combining of the constraints is increasing, i.e.,
%$\forall j \in I. a_j \leq \bigotimes_{i \in I} a_i$.
Thanks to distributivity, we can show that
$\otimes$ is monotone, and since $\bot$ is the identity of the monoid,
monotonicity implies that the combination of constraints is increasing, i.e.,
$\forall a, b \in A. a \leq a \otimes b$ holds.
%
Finally, we recall that by definition $\bigwedge \emptyset = \top$, so that 
$\forall a \in A. a \otimes  \top = \top$ also holds.\footnote{A symmetric 
choice $\langle A, \otimes, \top \rangle$
with distributivity with respect to $\bigvee$ (and thus
$a \otimes \bot = \bot$)
is possible: the monoidal operator would be decreasing, so that for
example $a \otimes b \leq a$. Indeed, this is the usual order in the
semiring-based approach to soft constraints~\cite{ecai06}.}
%
%$\bigotimes \emptyset = \bigwedge \emptyset$ $( = \top)$

%\smallskip
In the following, we fix a CLIM ${\mathbb S} = \langle A, \leq, \otimes \rangle$.
%
The next step is to provide an infinite composition. Our definition is from~\cite{CLIM}
(see also~\cite[p.~42]{golan}).

\begin{definition}[Infinite composition]
Let $I$ be a (countable) set of indexes. Then, the composition $\bigotimes_{i \in I} a_i$
is defined as $\bigvee_{J \subseteq I} \bigotimes_{j \in J} a_j$ for all finite subsets $J$.
\end{definition}

Should $I$ be finite, the definition gives back the usual multi-set composition,
since $\otimes$ is monotone and increasing.
%
As the infinitary composition is also monotone and increasing, by construction 
%$\bigotimes \emptyset = \bigvee \emptyset = \bot$ and 
$\bigotimes A = \bigvee A = \top$.
%
We now provide a notion of compactness with respect to the
monoidal operator.
% via the infinite composition.

\begin{definition}[$\otimes$-compact elements]\label{def:compactness}
Let $a \in A$. It is $\otimes$-compact 
(or $\otimes$-finite) 
if whenever $a \leq \bigotimes_{i \in I} a_i$
then there exists a finite subset $J \subseteq I$ such that $a \leq \bigotimes_{j \in J} a_j$.
\end{definition}

We let $A^\otimes \subseteq A$ denote the set of $\otimes$-compact elements of ${\mathbb S}$.
%
It is easy to show that a compact element is also $\otimes$-compact, i.e. $A^C \subseteq A^\otimes$.
Indeed, the latter notion is definitively more flexible.

\begin{example}
%
Consider the CLIM $\langle [0, 1], \geq, \times \rangle$ examined above, which corresponds to  
the segment of the reals with the inverse of the usual order and multiplication 
as monoidal product.
Since any infinite multiplication tends to $0$, then all the elements are
$\otimes$-compact, except the top element itself, that is, precisely $0$.
\end{example}
%
%Instead of the LUB, an alternative monoidal product is multiplication.
%%
%Since any infinite multiplication tends to $0$, then all the elements are
%$\otimes$-compact, except the top element itself, that is, precisely $0$.
%\marginpar{is $\otimes$-algebraicity needed?}

\begin{remark}
%Requiring that $\otimes$ is idempotent means that $a \otimes a = a$ for all elements.
%
It is easy to show that idempotency implies that $\bigotimes$ 
%is now defined on sets, and 
%moreover, that it 
coincides with LUBs, that is,
$\bigotimes S = \bigvee S$ for all subsets $S \subseteq A$.
%\marginpar{check infinite ones}
%
In other words, the whole soft structure collapses to a
complete distributive lattice.
%
Indeed, requiring distributivity makes the soft paradigm not fully comparable with the crisp one.
We are going to discuss it again in the concluding remarks.
\end{remark}

\subsection{Residuation}
Our first operator on CL-enriched IMs
%The first 
is a simple construction for a weak inverse of the
monoidal operator in CL-enriched monoids, known in the literature
on enriched monoid as residuation~\cite{golan,residuated}.
% residuation is a standard technique in enriched monoids~\cite{golan}, and
%it can be easily derived for CL-enrichment.

\begin{definition}[Residuation]
Let $a, b \in A$. The residuation of $a$ with respect to $b$ is defined as
$a \odiv b = \bigwedge \{ c \in A \mid a\leq b \otimes c\}$.
%\marginpar{check $a$ comp. implies $a \odiv b $ comp.}
\end{definition}

The definition conveys the intuitive meaning of a division operator: indeed,
$a \leq b \otimes (a \odiv b)$, thanks to distributivity.
%
Also,
$(a \otimes b) \odiv b \leq a$ and $a \odiv (b \otimes c) = (a \odiv b) \odiv c$.
%should $\otimes$ be idempotent, $b \leq a$ implies $a \odiv b = a$.
%
Residuation is monotone on the first argument: if $a \leq b$ then $a \odiv c \leq b \odiv c$
and 
%if $a \leq b$, then 
$a \odiv b = \bot$.
%
For more properties of residuation we refer to~\cite[Table~4.1]{resbook}.
%Here. we just note that it preserves the LUBs:
%$(\bigvee A) \odiv b = \bigvee_{a \in A} a \odiv b$.
%\marginpar{check $\leq$}


Most important for our formalism is the following result on $\otimes$-compactness.
%, which extends compactness to residuation for the first time.

\begin{lemma}
\label{preres}
Let $a, b \in A$. If $a$ is $\otimes$-compact, so is $a \odiv b$.
\end{lemma}

\begin{proof}%[of Lemma~\ref{preres}]
	If $a \odiv b \leq \bigotimes_{i\in I} a_i$, then by monotonicity
	$a \leq \bigotimes_{i \in I \uplus \{*\} } a_i$ for $a_* = b$.
	By $\otimes$-compactness of $a$ there exists a finite $J \subseteq I$
	such that $a \leq \bigotimes_{j \in J \uplus \{*\} } a_j$, and by the definition
	of division $a \odiv b \leq \bigotimes_{j \in J} a_j$, hence the result holds.
\end{proof}

Most standard soft instances (boolean, fuzzy, probabilistic, weighted, and so on)
%(i.e.\emph{Classical} $\langle \{\mathit{true},\mathit{false}\}, \vee, \wedge, \mathit{true}, \mathit{false}\rangle$, 
%\emph{Fuzzy} $\langle [0..1], \mathit{max}, \mathit{min}, 1, 0 \rangle$, \emph{Probabilistic} $\langle [0..1], \mathit{max}, \mathit{times}, 1, 0 \rangle$
%and \emph{Weighted CSPs} $\langle \mathbb{R}^+ \cup\{+\infty\}, min, \mathit{sum}, 0, +\infty \rangle$) 
are described by CL-enriched monoids and are  residuated: see e.g.~\cite{ecai06}.
% ($\mathit{times}$ and $\mathit{sum}$ are the corresponding arithmetic operations).
%
For these instances the $\odiv$ operator is used to (partially) remove constraints from the store,
and as such is going to be used in Section~\ref{sec:4}.
%\marginpar{not in the ex.1 so far}
%
In fact, in the soft literature it is usually required a tighter relation of (full) invertibility, 
also satisfied by all the previous CLIMs instances,
%that $b \otimes (a \odiv b) = a$ for $b \leq a$,
%in order to have what are called local consistency algorithms 
%that preserves the solution of
%a problem~\cite{ecai06}.
stated in our framework by the definition below.
%Yet, in our presentation we can relax this condition.
%\marginpar{maybe not: we'll see}

\begin{definition}
${\mathbb S}$ is invertible if
%and preserves $\otimes$-compactness if
%\begin{itemize}
%\item $\forall a, b \in A. 
$b \leq a$ implies $b \otimes (a \odiv b) = a$
for all $a, b \in A^\otimes$.
%\item $\forall a, b \in A. a, b \in C^\otimes$ implies $(a \odiv b) \in C^\otimes$
%\end{itemize}
\end{definition}

%In the following, we assume that our CLIM is invertible.
%% and preserves $\otimes$-compactness.
%%\marginpar{Maybe, item 2 can be derived}

\subsection{Cylindrification}
We now consider two families of operators for modellng
the hiding of variables and the passing of parameters in soft CCP.
%
They can be considered as generalised notions of existential quantifier
and diagonal element~\cite{popl91}, which are expressed in terms of operators 
of cylindric algebras~\cite{monk}.~\footnote{However, since we consider monoids instead of groups, 
the set of axiom of diagonal operators is included in the standard one for cylindric algebras.}


\begin{definition}[Cylindrification]
Let $V$ be a set of variables.
%
A cylindric operator $\exists$ over ${\mathbb S}$ and $V$ is a family of monotone, 
$\otimes$-compactness preserving functions
$\exists_x: A \rightarrow A$ indexed by elements in $V$ such that for all $a, b \in A$ and $x, y \in V$
\begin{enumerate}
 \item $\exists_x a \leq a$;
 \item $\exists_x(a \otimes \exists_x b) = \exists_x a \otimes \exists_x b$; 
 \item $\exists_x\exists_y a = \exists_y\exists_x a$.
 \end{enumerate}

\noindent Let $a \in A$. The \emph{support} of $a$ is the set of variables $sv(a) = \{ x \in V \mid \exists_x a \neq a\}$. 
% and the set of unsupported variables of $a$ is the set of variables $uv(a) =  V \setminus sv(a)$.
\end{definition}

%In the following, we will often refer to the elements in the support of $a$ as the \emph{free} variables of $a$.
%
%We will often use the definition of support in Section~\ref{sec:detSCCP} and Section~\ref{sec:ltsSCCP}, together with the definition of free variables.

%Each function of the cylindric operator is contractive and idempotent.
%
%Given their commutativity, 
For a finite $X \subseteq V$, let
$\exists_X a$ denote any sequence of function applications.
%
%now part of the definition
\remove{
\begin{lemma}
Let $a \in A$, $V$ a set of variables, $x \in V$ and $\exists$ a cylindric operator over ${\mathbb S}$ and $V$.
If $a$ is $\otimes$-compact, so is $\exists_x a$.
\end{lemma}
%\marginpar{da provare}

\begin{proof}
%TO DO

Let us assume that $\exists_x a \leq \bigotimes_{i \in I} a_i$ we need to prove that there is a finite set $J$ s.t.
$J \subseteq I$ and $\exists_x a \leq \bigotimes_{j \in J} a_j$.
First let $a_* = a \odiv \exists_x a$ now by monotonicity and our hypothesis 
we have $(\exists_x a) \otimes a_* \leq \bigotimes_{i \in I \uplus \{*\}} a_i$,
now by definition of $\odiv$ we know that $(\exists_x a) \otimes a_* = (\exists_x a) \otimes (a \odiv \exists_x a) \geq a$,
thus $a \leq \bigotimes_{i \in I \uplus \{*\}} a_i$.

Since $a$ is $\otimes$-compact then there exists $J$ s.t. $J \subseteq I$ and $a \leq \bigotimes_{j \in J \uplus \{*\}} a_j$,
namely $a \leq \bigotimes_{j \in J} a_j \otimes a_*$.
Then by monotonicity $a \odiv a_* \leq (\bigotimes_{j \in J} a_j \otimes a_*) \odiv a_*$ in addition 
$(\bigotimes_{j \in J} a_j \otimes a_*) \odiv a_* \leq \bigotimes_{j \in J} a_j$ by definition of $\odiv$,
hence $a \odiv a_* \leq \bigotimes_{j \in J} a_j$.
Now notice that $a \odiv a_* = a \odiv (a \odiv \exists_x a)$ and by definition of $\odiv$ we get
$a \odiv (a \odiv \exists_x a) \geq (a \odiv \exists_x a) \otimes \exists_x a \geq \exists_x a$,
therefore since $a \odiv a_* \geq \exists_x a$ then we can conclude that $\exists_x a \leq \bigotimes_{j \in J} a_j$
as wanted. \qed
\end{proof}
}
%
%In the following, w
We now fix a set of variables $V$ and a cylindric operator $\exists$ over ${\mathbb S}$ and $V$.
%~\footnote{We are not going to need the other standard component proposed in the literature , i.e., \emph{diagonals}: a %family of elements $d_{x, y} \in A$ indexed by pairs of elements in $V$.}

\begin{definition}[Diagonalisation]\label{def:diagonalisation}
A diagonal operator $\delta$ for $\exists$ is
a family of idempotent elements
$\delta_{x, y} \in A$ indexed by pairs of elements in $V$ such that
$\delta_{x, y} = \delta_{y, x}$ and
%$sv(\delta_{x, y}) = \{x,y\}$ for $x \neq y$ and
%
for all $a \in A$ and $x, y, z \in V$
%\marginpar{more axioms?}
 \begin{enumerate}
 \item $\delta_{x,x} = \bot$;
 \item if $z \not \in \{x, y\}$ then $\delta_{x,y} = \exists_z (\delta_{x,z} \otimes \delta_{z,y})$;
 \item if $x \not= y$ then $a \leq \delta_{x,y} \otimes \exists_x (a \otimes \delta_{x,y})$.
\end{enumerate}
\end{definition}

Axioms $1$ and $2$ above plus idempotency imply that $\exists_x \delta_{x,y} = \bot$,
which in turn implies (by axiom $2$ and idempotency of $\exists$) that $sv(\delta_{x,y}) = \{x,y\}$ for $x \neq y$.
%With respect to the literature, we added an axiom: the right equation in item~1.
%
Diagonal operators are used for modeling
variable substitution and parameter passing.
%
In the following, we fix a diagonal operator $\delta$ for $\exists$.

\begin{definition}[Substitution]
\label{def:sub}
Let $x, y \in V$ and $a \in A$.
%
The substitution $a[^y/_x]$ is defined as $a$ if $x = y$ and as
$\exists_x(\delta_{x,y} \otimes a)$ otherwise.
\end{definition}

Substitution behaves correctly with respect to $\exists$.

\begin{lemma}
\label{lemmaSubs0}
Let $x, y, w \in V$ and $a \in A$. Then it holds
\begin{itemize}
\item $(\exists_x a) [^y/_x] = a$
\item $\exists_x a = \exists_y (a [^y/_x])$ for $y \not \in sv(a)$
\item $(\exists_w a) [^y/_x] =  \exists_w (a [^y/_x])$ if $w \not \in \{x, y\}$
\end{itemize}
\end{lemma}

Finally, we rephrase some further laws of the crisp case 
%It is easy to prove that the following laws hold 
(see~\cite[p.140]{pippo}).

\begin{lemma}
\label{lemmaSubs}
Let $x, y \in V$ and $a \in A$. Then it holds
\begin{enumerate}
\item $(a[^y/_x])[^x/_y] = a$ for $y \not \in sv(a)$
\item $a[^y/_x] \otimes b[^y/_x] = (a \otimes b)[^y/_x]$
\item $x \not \in sv(a[^y/_x])$.
\end{enumerate}
\end{lemma}
\begin{proof}
	Consider e.g. the most difficult item $2$.
	By definition $a[^y/_x] \otimes b[^y/_x] = \exists_x (\delta_{x,y} \otimes a) \otimes \exists_x (\delta_{x,y} \otimes b)$, which in turn coincides with $\exists_x (\delta_{x,y} \otimes a \otimes \exists_x (\delta_{x,y} \otimes b))$ by axiom $2$ of $\exists$; by axiom $3$ of $\delta_{x,y}$ we have that $(a \otimes b) [^y/_x] = \exists_x (\delta_{x,y} \otimes a \otimes b) \leq \exists_x (\delta_{x,y} \otimes a \otimes \exists_x (\delta_{x,y} \otimes b))$, while the vice versa holds by the monotonicity of $\exists_x$.
\end{proof}



\end{document}

%\subsection{Possibly useful stuff}
 %
%\begin{lemma}
%label{undercomp}
%Let $a$, $b \in A$ such that $a \leq b$. If $b$ is $\otimes$-compact, so is $a$.
%\end{lemma}
%\begin{proof}
%TO DO. Needed in Lemma~\ref{comp}.
%\end{proof}
%\marginpar{To keep or to cut?}
%
% \begin{lemma}
% \label{preserve}
% If $\sigma_1$ and $\exists_x\sigma_2$ are $\otimes$-compact, then $\exists_x(\sigma_1 \otimes \sigma_2)$
% is so.
 %\end{lemma}
 %\begin{proof}
%TO DO. Needed in Lemma~\ref{comp}.
%\ \end{proof}
 %
  %\begin{lemma}
 %\label{preserve2}
 %If $\exists_x\sigma_2$ is $\otimes$-compact, then $\exists_x(\sigma_1 \odiv \exists_x \sigma_2)  = \exists_x \sigma_1 \odiv \exists \sigma_2$. 
 %is so.
 %end{lemma}
 %\begin{proof}
%TO DO. Needed in Lemma~\ref{comp}.
% \end{proof}
