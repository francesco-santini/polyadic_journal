\documentclass[main.tex]{subfiles}
\begin{document}

\section{Deterministic Soft CCP}\label{sec:detSCCP}
We now introduce our language. 
%In the following we
We fix an invertible CLIM $\mathbb S = \langle {\mathcal C},$ $\leq, \otimes\rangle$, which is also cylindric 
over a set of variables $V$, denoting by
$c$ an element in ${\mathcal C}^\otimes$.
%
%We now recall the syntax of (the deterministic fragment) of soft CCP
\[ A \Coloneqq \: \: \mathit{\ostop} \mid \textit{\tell}(c)  \mid \textit{\ask}(c) \rightarrow A \mid A \parallel A \mid \exists_x A \mid %Z \mid \mu_Z A
p(x).\]  
%
%
%%Given a soft constraint system $S_C=\langle \C, \otimes,\bot,\top ,
%%\exists_x, d_{xy}\rangle$ as defined in
%%Section~\ref{sec:softconstraints}, the syntax
%%of  deterministic \SCCP agents is defined as
%
%
%\def\odivv{\; {\ominus\hspace{-4pt} \div} \;}
%\def\odivvv{\; {\ominus\hspace{-6pt} \div} \;}
%\begin{table}
%%\footnotesize
%\begin{center}
%%P \Coloneqq \: \: F.A \; \; \; \; \; \; \; \;
%%F \Coloneqq \: \: p(\hat{x})::A \mid F.F \; \; \; \; \; \; \; \;
%$A \Coloneqq \: \: \mathit{\ostop} \mid \textit{\tell}(c)  \mid \textit{\ask}(c) \rightarrow A \mid A \parallel A \mid \exists_x A \mid %Z \mid \mu_Z A$
%p(x)$
%\end{center}
%%\normalsize
%\caption{The syntax of deterministic soft CCP}
%\label{syntax}
%\end{table}
%
%In Tab.~\ref{syntax} we recall the deterministic fragment of soft CCP,
%with respect to the standard proposal, we replaced the procedure call with the
%simpler to handle recursion operator $\mu_Z$.
%%\marginpar{A che servono F e P?}
%%
Let $\mathcal{A}$ be the set of all agents, which is
parametric with respect to a set $\mathcal{P}$ of (unary) procedure declarations 
$p(x) = A$ such that $fv(A) = \{x\}.$\footnote{The set of free variables of an agent is defined in the expected way by structural induction, assuming that $fv(\tell(c)) = sv(c)$ and $fv(\ask(c) \rightarrow A) = sv(c) \cup fv(A)$.}
%
%$\mathcal{P}$ is the class of programs, $\mathcal{F}$ is the class of sequences of procedure
%declarations,
%$\mathcal{A}$ is the set of agents and
%$\mathcal{Z} = \{Z, Z_1, Z_2 \ldots\}$ is the set of recursion variables, 
%$c \in \C^\otimes$ with $sv(c) \in V$.
%
%, and $\hat{x}$ is a tuple of variables in $Var$ passed in a procedure call.
%
%In $p(\hat{x}) :: A$ we assume that the variables in
%$\mathit{vars}(A)$, i.e. the set of variables occurring either as parameters of procedure calls or belonging
%to the support of
%the constraints in agent $A$, also occur
%among $\hat{x}$.
%\footnote{Note that, with respect to the standard proposals, we require only for the
%$\textit{ask}$ operator to have a compact constraint, while the $\textit{tell}$ ranges over all constraints.
%Intuitively, we allow to require to the store the removal of certain constraint, while compactness allow
%for obtaining a simpler relation when checking the entailment of a constraint.}

In Table~\ref{fig:operational} we provide a reduction semantics for SCCP:
%
%It is defined as a triple $\langle \Gamma, T,  \rightarrow
%\rangle$, where $\Gamma = {\mathcal A} \times {\mathcal C}$ is the set of all possible configurations,
%$T = \{\mathit{stop}\} \times {\mathcal C} $ is the set of \emph{terminal} configurations and
%$\rrarrow \, \, \subseteq \, \,\Gamma \times   \Gamma$ is a binary relation between
%configurations.
%It is defined as
a pair $\langle \Gamma,  \rightarrow \rangle$, for $\Gamma = {\mathcal A} \times \C^\otimes$
the set of configurations  and $\rrarrow \, \, \subseteq \, \,\Gamma \times   \Gamma$ a family 
 of binary relations indexed over set of variables,
%$2^V$,
%between them, 
i.e., $\rrarrow = \bigcup_{\Delta \subseteq V} \rrarrow_\Delta$ and 
$\rrarrow_\Delta \, \, \subseteq \, \,\Gamma \times \Gamma$.
%indexed by subset of variables.

%\vspace{-.25cm}
\def\odiv{\; {\ominus\hspace{-6pt} \div} \;}
\def\odivvv{\; {\ominus\hspace{-6pt} \div} \;}
\begin{table}  %\hfil5
  
  %\scalebox{0.9}{
   \begin{center}
   \begin{tabular}{llll} 
   %
   \mbox{\bf R1}& $\frac{\displaystyle sv(\sigma) \cup sv(c) \subseteq \Delta } {\displaystyle \langle \hbox{\tell}(c), \sigma \rangle \rrarrow_\Delta  \langle \hbox{\ostop},
                                               \sigma \otimes c\rangle}$
   \ \ \ & \bf{Tell}&
  \\ 
  &\mbox{   }&\mbox{   } &\mbox{   }
  \\
  \mbox{\bf R2}& $\frac {\displaystyle c \leq \sigma \wedge  sv(\sigma) \cup sv(c) \subseteq \Delta}{\displaystyle
  	\begin{array}{l} \langle \hbox{\ask}(c) \rightarrow A, \sigma \rangle \rrarrow_\Delta \langle A, \sigma \rangle   	\end{array}}$
    \ \ \ & \bf{Ask}&
    \\
    &\mbox{   }&\mbox{   }&
    \\
  \mbox{\bf R3}& $\frac {\displaystyle \langle A,\sigma \rangle \rrarrow_\Delta \langle A', \sigma' \rangle
  \wedge fv(B) \subseteq \Delta} 
  {\displaystyle \begin{array}{l}
                          \langle A\parallel B, \sigma \rangle \rrarrow_\Delta \langle A'\parallel B, \sigma' \rangle
                          \end{array}}$ 
    & \bf{Par1}&
  \\
  & \mbox{   }&\mbox{   }&
  \\
    \mbox{\bf R4}& $\frac {\displaystyle \langle A,\sigma \rangle \rrarrow_\Delta \langle A', \sigma'   \rangle
    	\wedge fv(B) \subseteq \Delta} 
    {\displaystyle 
    	\begin{array}{l} \langle B\parallel A, \sigma \rangle \rrarrow_\Delta \langle B\parallel A', \sigma' \rangle
    	\end{array}}$& \bf{Par2}&
    \\
    &\mbox{   }&\mbox{   }&
    \\
  \mbox{\bf R5}& $\frac {\displaystyle \{y\} \cup sv(\sigma) \subseteq \Delta \wedge \displaystyle p(x) = A \in  \mathcal{P} }
  {\displaystyle\langle p(y),\sigma\rangle \rrarrow_\Delta \langle  A[^y/_x], \sigma\rangle}$ 
  &\bf{Rec}&
    \\
   &\mbox{   }&\mbox{   }&
  \\
    \mbox{\bf R6}& $\frac {\displaystyle fv(A) \cup sv(\sigma) \subseteq \Delta \wedge w \not \in \Delta }
    {\displaystyle\langle \exists_x A,\sigma\rangle \rrarrow_\Delta \langle A[^w/_x], \sigma\rangle}$
    &\bf{Hide}&
  \end{tabular}
  \end{center}
\caption{Reduction semantics for SCCP.}
\label{fig:operational}
\end{table}
\def\odiv{\, {\ominus\hspace{-7.8pt} \div} \,}
\def\odivvv{\; {\ominus\hspace{-4.7pt} \div} \;}


In {\bf R1} a constraint $c$ is added to the store $\sigma$.
%, which in the next step will be $\sigma \otimes c$.
%
{\bf R2} checks if $c$ is entailed by  $\sigma$: if not, the computation is blocked.
% until this condition holds, and then continuation $A$ is executed.
%
Rules {\bf  R3}  and {\bf  R4} model the interleaving of two agents in parallel.
%provides the standard unfolding step for the recursion operator: the mechanisms of variable.
%substitution and scope are defined as usual.
%
%Rules {\bf  R3}  and {\bf  R4} model the interleaving of two agents in parallel.
%% symmetric rules (on agent $B$) are not shown in Fig.~\ref{fig:operational}.
%\marginpar{Either $\exists_x \ostop = \ostop$ is missing or R4 is basically useless}
%
Rule {\bf R5} replaces a procedure identifier with the associated body, renaming the formal parameter with the actual one:
$A[^y/_x]$ stands for the agent obtained by replacing all the occurrences of $x$ with $y$.\footnote{With the usual 
conventions, so that e.g. $(\exists _y A)[^y/_x] = \exists_w ((A[^w/_y])[^y/_x])$ for $w \not \in sv(A) \cup \{x,y\}$
and $\hbox{\tell}(c)[^y/_x] =  \hbox{\tell}(c[^y/_x])$, the latter defined according to Definition~\ref{def:sub}.}
%
Rule {\bf R6} hides the variable $x$ occurring in $A$. 
%
The variable $w$ that replaces $x$ is globally fresh,
as ensured by requiring $w \not \in \Delta$.
The latter is more general than just requiring that 
$w \not \in fv(A) \cup sv(\sigma)$, since
$\langle B, \rho \rangle   \rrarrow_\Delta$ implies that $fv(B) \cup sv(\rho) \subseteq \Delta$.\footnote{Our rule is  reminiscent of $(8)$ in~\cite[p.~342]{popl91}.}
%
%
%(as $(9)$ in~\cite{popl91}), assuming a set of variables $Var_{\mathcal{P}}$
%(one for each procedure definition) that cannot occurr either in the support nor in the syntax of any agent.
%
%\marginpar{R5 may add non-determinsm!!}
%\marginpar{R5 is much more difficult now: the problem was in the superscript $\sigma''$ for non-idempotent CLIMs}
%Finally, rule {\bf R6} calls a procedure $p(\hat{x})$ by passing the (global) parameters $\hat{y}$:
%%$d_{\hat{x},\hat{y}}$ is a shorthand for the composition of the $d_{x_i,y_i}$'s.
%$A[\hat{y}/\hat{x}]$ is the pairwise renaming in $A$ of variables in  $\hat{y}$ with variables in $\hat{x}$.
%\marginpar{The use of $d_{x,y}$ for R6 would be nicer, but maybe problematic (compactness)}
%Finally, rule {\bf R6} provides the standard unfolding step for the recursion operator: the mechanisms of variable
%substitution and scope are defined as usual.

%\smallskip
We denote $fv(A) \cup sv(\sigma)$ as $fv(\gamma)$ for a configuration
$\gamma = \langle A, \sigma \rangle$, and by
$\gamma[^z/_w]$ the component-wise application of substitution $[^z/_w]$.
%
 Clearly $\gamma \rightarrow_\Delta \gamma'$ 
 implies $fv(\gamma) \subseteq \Delta$, and 
 %
 we now further provide three lemmata on reduction.
 %, which are needed later on.

\begin{lemma}[Mono]\label{mono}
Let $\langle A, \sigma \rangle \rightarrow_\Delta \langle B, \sigma' \rangle$
be a reduction. Then, $\sigma \leq \sigma'$ and $sv(\sigma') \subseteq \Delta$.
\end{lemma}

The proof is straightforward: only rule {\bf R1} can modify the store,
and $\sigma \leq \sigma \otimes c$ as well as 
$sv(\sigma \otimes c) \subseteq sv(\sigma) \cup sv(c)$ hold,
since as shown above $fv( \hbox{\tell}(c) ) \cup sv(\sigma) \subseteq \Delta$.
%%The difficult case is  {\bf  R7}, 
%%which is solved since $\sigma \leq  \sigma_0 \otimes \exists_x \sigma'$ and by monotonicity 
%%of $\exists_x$ and $- \odiv \exists_x \sigma_0$. 
%%
%Note that 
%%$A$ may be an extended agent, and 
%$\sigma$ is not necessarily $\otimes$-compact.

%A simple fact to establish is that if 
%$\langle A, \sigma \rangle \rightarrow \langle \exists_x^{\sigma'} B, \sigma" \rangle$
%is a reduction, then $\exists_x \sigma' \leq \sigma"$.
%More can be said by restraining $A$ and $\sigma$. 
%
%\begin{definition}[extended agents, admissible configurations]
%An extended agent $A \in \mathcal{A}^+$ is either an agent $A \in \mathcal{A}$ or it has the shape 
%$A = A_1 \parallel A_2$ and both $A_i$'s are extended agents or it has the shape 
%$\exists_x^\rho A_0$ and $A_0$ is an extend agent and $\exists_x \rho \in \mathcal{C}^\otimes$.
%
%A configuration $\langle A, \sigma \rangle$ is admissible if $A \in \mathcal{A}^+$ and if its shape is 
%$A = A_1 \parallel A_2$ then both $\langle A_i, \sigma \rangle$ are admissible or if its shape is
%$\exists_x^\rho A_0$ then $\langle A_0, \rho \otimes \exists_x (\sigma \odiv \exists \rho) \rangle$ is admissible.
%\end{definition}

%So, let $\mathcal{A}^+$ denote the set of agents obtained by including 
%the extended operators $\exists_x^{\sigma'} -$ such that $\exists_x\sigma'$ is $\otimes$-compact, and let 
%$\langle A, \sigma \rangle$ be admissible if $A \in \mathcal{A}^+$ and whenever $A = A_0 \parallel \ldots \parallel A_n$ %such that $A_i =\exists_x^{\sigma'} A'_i$ then $\exists \sigma' \leq \sigma$ and $\langle A'_i, \sigma' \otimes \exists_x 
%(\sigma \odiv \exists \sigma') \rangle$ is admissible.

%\begin{lemma}\label{comp}
%Let $\langle A, \sigma \rangle \rightarrow \langle B, \sigma' \rangle$ be
%a reduction. If $\langle A, \sigma \rangle$ is admissible, then $\langle B, \sigma' \rangle$ is so and $\sigma' \odiv \sigma \in \mathcal{C}^\otimes$.
%\end{lemma}
%\begin{proof}
%[TO FINISH]
%We proceed by induction on the length of the proof. 
%The axioms {\bf R2}, {\bf R5}, and {\bf R6} are straightforwardly checked, so, the only one 
%to be verified is if the last inference rule applied 
%%in the $n$-th step 
%is {\bf R7}.
%%
%So, let $\langle \exists^{\sigma'}_x A,
%\sigma \rangle \rrarrow \langle \exists^{\sigma_1}_x B, \sigma_0 \otimes \exists_x \sigma_1\rangle$ with the construction of rule {\bf R7}. By hypothesis, $up = \sigma'' \odiv (\sigma' \otimes \exists_x \sigma_0) =  (\sigma'' \odiv \sigma') \odiv \exists_x \sigma_0 = \sigma_1 \odiv \sigma'$ is $\otimes$-compact. Now, $\sigma_1 \leq up \otimes \sigma'$, so that 
%$\exists_x \sigma_1 \leq \exists_x (up \otimes \sigma')$: by Lemma~\ref{preserve} $\exists_x (up \otimes \sigma')$ is 
%$\otimes$-compact, and by Lemma\ref{undercomp} so is $\exists_x \sigma_1$, hence $\exists_x^{\sigma_1} B \in \mathcal{A}^+$.
%
%Since $\langle B, \sigma''\rangle$ is admissible, and $\sigma_1 \otimes \exists_x [(\sigma_0 \otimes \exists_x \sigma_1) \odiv \exists_x \sigma_1)]$
%
%Now, consider $up' = (\sigma_0 \otimes \exists_x \sigma_1) \odiv \sigma$. By hypothesis we have that 
%$\sigma = \sigma_0 \otimes \exists_x \sigma'$, thus 
%$up' = (\sigma_0 \otimes \exists_x \sigma_1) \odiv (\sigma_0 \otimes \exists_x \sigma') = 
%[(\sigma_0 \otimes \exists_x \sigma_1) \odiv \sigma_0] \odiv \exists_x \sigma') \leq \exists_x \sigma_1 \odiv \exists_x\sigma'$, 
%and since $\exists_x \sigma_1$ is $\otimes$-compact, so is $\exists_x \sigma_1 \odiv \exists_x \sigma'$ 
%by Lemma~\ref{preres} and thus $up'$ by Lemma\ref{undercomp}.
%\end{proof}

%We now establish two important results concerning computations.
%
%\begin{proposition}
%Let $\langle A, \sigma \rangle \rightarrow^* \langle B, \sigma' \rangle$
%be a computation. If $A \in \mathcal{A}$ and $\sigma \in {\mathcal C}^\otimes$, then
%$B \in \mathcal{A}^+$, $\sigma' \in {\mathcal C}^\otimes$ and $\sigma \leq \sigma'$.
 %\end{proposition}
%
%We now establish an additional result on the operational monotonicity.
% but first, we say that for a reduction $\xi: 
%The proposition is an immediate consequence of the lemmata above.

\begin{lemma}[Operational mono]\label{opmonotonicity}
Let $\langle A, \sigma \rangle \rightarrow_\Delta \langle B, \sigma' \rangle$
be a reduction and $\rho \in  \C^\otimes$ such that $sv(\rho) \subseteq \Delta$. Then,
 there exists a reduction $\langle A, \sigma \otimes \rho \rangle \rightarrow_\Delta \langle B, \sigma' \otimes \rho \rangle$.
\end{lemma}

The proof is straightforward, since as before $sv(\sigma \otimes \rho) \subseteq sv(\sigma) \cup sv(\rho)$ and 
moreover $\sigma, \rho \in  \C^\otimes$ ensure that $\sigma \otimes \rho \in  \C^\otimes$.

\subsection{Observational Semantics}
To define fair computations (Definition~\ref{def:fair}), we introduce enabled and active agents.
Note that any transition is generated by an agent of the shape $\mathit{\tell}(c)$ or  $\mathit{\ask}(c) \rightarrow A$ 
or $p(x)$ or $\exists_x A$ via the application of precisely one instance of one of the axioms {\bf R1}, {\bf R2},  {\bf R5}, and {\bf R6} of Table~\ref{fig:operational}.
%
An agent of such  shape is \emph{active} in a transition $t = \gamma \rightarrow \gamma'$ if it generates such transition, i.e. if there is a derivation of $t$ where that agent is used in the building axiom.
%
Moreover, an agent is \emph{enabled} in a configuration $\gamma$ if there is a transition $\gamma \rightarrow \gamma'$ such that the agent is \emph{active} in it.
%\marginpar{qua ci vogliono gli agenti, non i processi}

\begin{definition}[Fair computations]\label{def:fair}
Let $\gamma_0  \rightarrow_{\Delta_1} \gamma_1  \rightarrow_{\Delta_2} \gamma_2 \rightarrow_{\Delta_3} \dots$ be a
(possibly infinite) computation. 
%\marginpar{added ``infinite'' before computation}
It is fair if it is increasing (i.e., $\Delta_k \subseteq \Delta_{k+1}$ for any $k$) and whenever an agent
$A$ is enabled in some $\gamma_i$ then $A$ is active in  $\gamma_j  \rightarrow_{\Delta_{j+1}} \gamma_{j+1}$ 
for some $j \geq i$.
\end{definition}

Note that fairness is well given: the format of the rules allows us to always trace the occurrence of an agent along a computation.
%\marginpar{R4 may be a problem for this tracing. Drop that rule!}

\begin{definition}[Observables]\label{def:observables}
Let $\xi = \gamma_0  \rightarrow_{\Delta_1} \gamma_1  \rightarrow_{\Delta_2} \dots$ be a (possibly infinite) computation with $\gamma_i = \langle A_i, \sigma_i\rangle$.
%
$\mathit{Result}(\xi)$ 
is $\bigvee_i (\exists_{X_i} \sigma_i)$, for $X_i = (fv(\gamma_i))\setminus(fv(\gamma_0))$.
%$\bigvee_{i} \sigma_i$.
\end{definition}

%We often denote $fv(A) \cup sv(\sigma)$ as $fv(\gamma)$ for a configuration
%$\gamma = \langle A, \sigma \rangle$.
%
%\smallskip

Similarly to crisp programming~\cite{popl91}, if a finite computation is fair
then it is deadlocked and 
%(and thus increasing),
%then it ends on a state $\langle \mathit{\ostop}, \sigma\rangle$, and
its result coincides with the store of the last configuration.
%\marginpar{intermediate stores are compact if tell are so}

\begin{proposition}[Confluence]\label{prop:confluence}
Let $\gamma$ be a configuration and $\xi_1$, $\xi_2$ two (possibly infinite)
computations of $\gamma$.
%
If $\xi_1$ and $\xi_2$ are fair, then $\mathit{Result}(\xi_1) = \mathit{Result}(\xi_2)$.
\end{proposition}


%As for crisp programming~\cite{popl91}, all the fair computations of a configuration have the same result.
%
%\begin{proposition}[confluence]\label{prop:confluence}
%Let $\gamma$ be a configuration and $\xi_1$, $\xi_2$ two fair computations of $\gamma$. Then $\mathit{Result}(\xi_1) = \mathit{Result}(\xi_2)$.
%\end{proposition}

The proposition is an immediate consequence of the lemma below.

\begin{lemma}\label{lemma:upto}
Let $\gamma \rightarrow_{\Delta_i} \gamma_i$ be reductions for $i =  1, 2$.
%such that $\gamma_1 \cong_\Delta \gamma_2$ and $\Delta \subseteq \Delta_1 \cup \Delta_2$.
Then one of the following holds
\begin{enumerate}
\item $\xi_i = \gamma \rightarrow_{\Delta_i} \gamma_i[^z/_{w_i}]$
and $\gamma_1[^z/_{w_1}] =  \gamma_2[^z/_{w_2}]$
for $w_i \not \in \Delta_i$ and $z$ fresh;
\item
$\xi_i = \gamma \rightarrow_{\Delta_i} \gamma_i[^{z_i}/_{w_i}] \rightarrow_{\Delta_1 \cup \Delta_2 \cup \{z_i\}} \gamma_3$
for $w_i \not \in (\Delta_1 \cap \Delta_2) \cup fv(\gamma_3)$ and $z_i$'s fresh.
\end{enumerate}
In both cases, $Result(\xi_1) = Result(\xi_2)$.
\end{lemma}

\begin{proof}%[of Lemma~\ref{lemma:upto}]
	%\marginpar{the proof is wrong}
	%Similarly to \cite{popl91}, Prop.~\ref{prop:confluence} can be proved by taking advantage of the commutativity of
	%transitions: if two different transitions $\langle A, \sigma \rangle \rightarrow \langle A_1, \sigma_1 \rangle$ and $\langle A, 
	%\sigma \rangle \rightarrow \langle A_2 , \sigma_2\rangle$ are enabled, then $\langle A_1, \sigma_1 \rangle \rightarrow 
	%\langle A_3, \sigma_1 \otimes \sigma_2\rangle$ and $\langle A_2, \sigma_2 \rangle \rightarrow \langle A_3, \sigma_1 
	%\otimes \sigma_2\rangle$ are possible.
	First of all, note that the calculus is deterministic except for the parallel and the hiding operators.
	Consider the latter. The problem may arise if different fresh variables are chosen, let us say $w_1$ and $w_2$.
	However, $\gamma_1[^z/_{w_1}] =  \gamma_2[^z/_{w_2}]$ 
	by replacing the new variables with a globally fresh one, as in item 1.
	
	So, let us assume that the two reductions occur on the opposite sides of a parallel operator.
	Also, let $\gamma \rightarrow_{\Delta_1} \gamma_1$ replace a hiding operator with a variable $w_1$
	(hence we have $w_1 \not \in \Delta_1$). 
	%
	If $w_1 \in fv(\gamma_2)$, since $w_1 \not \in \gamma$ and the only reduction enlarging the set of free variables is the 
	replacement of a hiding operator, also $\gamma \rightarrow_{\Delta_2} \gamma_2$ 
	must replace a hiding operator with variable $w_1$, and thus it suffices to replace $w_1$ 
	with fresh variables $z_1$ and $z_2$ in  the two reductions, in order for item 2 to be verified.
	%
	If $w_1 \not \in fv(\gamma_2)$, then $\xi_2$ is obtained by replacing in $\gamma_2$ 
	the hiding operator with $z_1$ instead of $w_1$. As for obtaining $\xi_1$, the only problematic case is 
	if  $\gamma \rightarrow_{\Delta_2} \gamma_2$ also replaces 
	a hiding operator with a variable $w_2 \in \Delta_1 \cup fv(\gamma_1)$.
	However, we have that $w_2 \not \in fv(\gamma_1)$ since otherwise (as shown above) $w_1 = w_2$, 
	thus $\xi_1$ is obtained by replacing in $\gamma_1$ the hiding operator with $z_2$ instead of $w_2$,
	and item 2 is then verified. 
		
	Among the remaining cases, the only relevant one is whenever both actions add different constraints to the store.
	%\medskip
	%
	So, let us assume that $\gamma = \langle A_1 \parallel A_2, \sigma \rangle$ such that 
	$\langle A_1, \sigma \rangle \rightarrow_{\Delta_1} \langle B_1, \sigma_1 \rangle$
	and
	$\langle A_2, \sigma \rangle \rightarrow_{\Delta_2} \langle B_2, \sigma_2 \rangle$.
	%Also, we may safely assume up-to that any locally fresh variable occurring in either $B_1$ or $B_2$ 
	%is globally fresh.
	%
	Note that since reduction semantics is monotone (Lemma~\ref{mono})
	and $\sigma$ is $\otimes$-compact, also $\sigma_1$
	is $\otimes$-compact and furthermore we have
	$\sigma_1 = \sigma \otimes (\sigma_1 \odiv \sigma)$.
	% (by Lemma~\ref{mono}).
	%
	Now, operational monotonicity (Lemma~\ref{opmonotonicity}) ensures us that
	$\langle B_1 \parallel A_2, \sigma \otimes (\sigma_1 \odiv \sigma) \rangle
	\rightarrow_{\Delta_1 \cup \Delta_2}
	\langle B_1 \parallel B_2, \sigma \otimes (\sigma_1 \odiv \sigma) \otimes (\sigma_2 \odiv \sigma) \rangle$ 
	and by symmetric reasoning the latter configuration 
	is the one we were looking for. \qed
\end{proof}


The result above is a local confluence theorem, which is expected, since the calculus is essentially deterministic.
The complex formulation is  due to the occurrence of hiding operators: as an example, different fresh variables may be chosen
for replacing $\exists_x$, such as  $w_1$ and $w_2$ in the first item above, and then a globally fresh variable $z$ has to 
be found for replacing them.

As a final remark, note that  $\gamma \rightarrow_\Delta \gamma'$ with $z \in fv(\gamma)$ and $w \not \in fv(\gamma')$
implies $\gamma[^w/_z] \rightarrow_{(\Delta\setminus \{z\}) \cup \{w\}} \gamma'[^w/_z]$.
Combined with the proposition above, they ensure that
%
fair computations originating from a configuration are  either all finite or all infinite, 
and furthermore  they have the same result.
%
So, in the following we denote as $\mathit{Result}(\langle A, \sigma \rangle)$ the unique result of the fair 
computations originating from $\langle A, \sigma \rangle$.
%
This fact allows to define an observation-wise equivalence.
% on agents.
% In Definition~\ref{def:obequivalence}, we show when two processes are observation-wise equivalent.

\begin{definition}[Observational equivalence]\label{def:obequivalence}
Let $A, B \in \mathcal {A}$ be agents. They are observationally equivalent ($A \sim_o B$) if 
$\mathit{Result}(\langle A, \sigma \rangle) = \mathit{Result}(\langle B, \sigma \rangle)$
for all $\sigma \in  \C^\otimes$.
%Let $\mathcal{O} : (\mathcal{A} \rrarrow \mathcal{C}^\otimes) \rrarrow \mathcal{C}$ be the function given by 
%$\mathcal{O}({A})(\sigma) = \mathit{Result}(\langle A, \sigma \rangle)$.
%We say that $A$ and $B$ are \emph{observationally equivalent} ($A \sim_o B$) if $\mathcal{O}(A) = \mathcal{O}(B)$.
\end{definition}

It is easily shown that $ \sim_o$ is preserved by all contexts, i.e., it is a \emph{congruence}.\footnote{Recall that a context $C[\bullet]$ is a syntactic expression with a single hole $\bullet$ 
such that replacing $\bullet$ with an agent $A$ in the context produces an agent, denoted by $C[A]$. For example if $C[\bullet]$ is the context $\tell(c) \parallel \bullet$  then $C[A] =  \tell(c) \parallel A.$ An equivalence $\cong$ between agents is a congruence if $A \cong B$ implies $C[A] \cong C[B]$ for every context $C[\bullet].$}


\subsection{Saturated Bisimulation}
As proposed in \cite{pippo} for crisp languages, we define a barbed equivalence between two agents~\cite{barbed}.  Since barbs are basic observations (predicates) on the states of a system, in this case they correspond to the compact constraints in $\mathcal{C}^\otimes$, and we say that $\langle A, \sigma \rangle$ verifies $c$, or that $\langle A, \sigma \rangle \downarrow_c$ holds, if  $c \leq \sigma$.
%
However, since \emph{barbed bisimilarity} is an equivalence already for CCP, along~\cite{pippo}
we propose the use of \emph{saturated bisimilarity}
%~\cite{barbedMontanari} has been proposed 
in order to obtain a congruence:
%
Definition~\ref{def:strongsb} and Definition~\ref{def:weaksb} respectively provide the strong and weak definition of saturated bisimilarity.
%We say that $\gamma = \langle P, \sigma\rangle$ satisfies the barb $c$, written $\gamma \downarrow_c$,
%iff $\gamma \longrightarrow \gamma'$ and $\gamma' \downarrow_c$.
%\marginpar{Are barbs compact?}

\begin{definition}[Saturated bisimilarity]\label{def:strongsb} A saturated bisimulation is a symmetric relation $R$ on configurations such that whenever
%$(\gamma_1,\gamma_2) \in R$ with $\gamma_1 = \langle A, \sigma \rangle$
%and $\gamma_2 = \langle B, \rho \rangle$
$( \langle A, \sigma \rangle,\langle B, \rho \rangle) \in R$
\begin{enumerate}
\item if $\langle A, \sigma \rangle \downarrow_c$ then $\langle B, \rho \rangle \downarrow_c$;
\item if $\langle A, \sigma \rangle \longrightarrow \gamma'_1$ then there exists $\gamma'_2$ such that $\langle B, \rho \rangle \longrightarrow \gamma'_2$ and $(\gamma'_1, \gamma'_2) \in R$;
\item $(\langle A,\sigma \otimes d\rangle, \langle B,\rho \otimes d \rangle) \in R$ for  all $d \in \mathcal{C}^\otimes$.
\end{enumerate}
We say that $\gamma_1$ and $\gamma_2$ are  saturated bisimilar ($\gamma_1  \sim_{\mathit{s}} \gamma_2$) if there exists a  saturated  bisimulation $R$ such that $(\gamma_1 , \gamma_2 ) \in R$. We write $A \sim_{\mathit{s}} B$ if $\langle A, \bot\rangle \sim_{\mathit{s}} \langle B, \bot \rangle$.
\end{definition}

We now let $\longrightarrow^*$ denote the reflexive and transitive closure of $\longrightarrow$, restricted to increasing computations.
We say that  $\gamma \Downarrow_c$ holds if there exists $\gamma' = \langle A, \sigma \rangle$ such that 
$\gamma \longrightarrow^* \gamma'$ and $c \leq \exists_{X} \sigma$ for $X = fv(\gamma') \setminus fv(\gamma)$.

\begin{definition}[Weak saturated bisimilarity]\label{def:weaksb}
Weak saturated bisimilarity
($\approx_{\mathit{s}}$) is obtained from Definition~\ref{def:strongsb} by replacing $\longrightarrow$ with $\longrightarrow^*$ and $\downarrow_c$ with $\Downarrow_c$.
\end{definition}

Since $\sim_{\mathit{s}}$ (and $\approx_{\mathit{s}}$) is itself a  saturated bisimulation, it is obvious that it is upward closed, and it is also a congruence with respect to all the contexts of SCCP (i.e., it is preserved under any context): indeed, a context $C[\bullet]$ can modify the behaviour of a configuration only by adding constraints to its store.
%\marginpar{to be proved for $\exists_x-$ (not needed for Prop.3)}

We now show that $\approx_{\mathit{s}}$, as given in Definition~\ref{def:weaksb}, coincides with the observational equivalence $\sim_o$ (see Definition~\ref{def:obequivalence}). First we recall the notion of and a classic result on \emph{cofinality}: two (possibly infinite) chains $c_0 \leq c_1 \leq \dots$ and  $d_0 \leq d_1 \leq \dots$ are said to be \emph{cofinal} if for all $c_i$ there exists a $d_j$ such that $c_i \leq d_j$ and, vice-versa, for all $d_i$ there exists a $c_j$ such that $d_i \leq c_j$.

\begin{lemma}\label{lem:cofinality} Let $c_0 \leq c_1 \leq \dots$ and $d_0 \leq d_1 \leq \dots $ be two chains. \emph{(1)} If they are cofinal, then they have the same limit, i.e., $\bigvee_i c_i = \bigvee_i d_i$. \emph{(2)} If the elements of the chains are $\otimes$-compact and $\bigvee_i c_i = \bigvee_i d_i$, then the two chains are cofinal.\end{lemma}
%\marginpar{questo richiede che tell inserisca solo compatti}
\begin{proof}
%[of Lemma~\ref{lem:cofinality}]
Let us tackle $(2)$, and consider the sequence $e_0 = c_0$ and $e_{i} = c_{i+1} \odiv c_i$.
Each $e_i$ is the difference between two consecutive elements of a chain.
%
Since the CLIM is invertible we have $c_k =  \bigotimes_{i \leq k} e_i$ and thus
$\bigvee_i c_i = \bigotimes_i e_i$. Since each $d_j$ is $\otimes$-compact and
$d_j \leq \bigotimes_i e_i$, there is a $k$ such that $d_j \leq \bigotimes_{i \leq k} e_i$.
The same reasoning is applied to the chain $d_0 \leq d_1 \leq \dots $, thus
the result holds. \qed
\end{proof}

For proving Proposition~\ref{prop:weaksbequivobs} we now relate weak barbs and fair computations.

\begin{lemma}\label{lem:barbsfair}
Let $\xi = \gamma_0 \longrightarrow \gamma_1 \longrightarrow \gamma_2 \longrightarrow \ldots$ be a (possibly infinite) fair computation. If $\gamma_0 \Downarrow_d$ then there exists a store $\sigma_i$ in $\xi$ such that $d \leq \exists_{X_i} \sigma_i$ for $X_i = fv(\gamma_i) \setminus fv(\gamma_0)$.
\end{lemma}

The lemma holds since the language is deterministic and computations fair.

\begin{proposition}\label{prop:weaksbequivobs}
$A \sim_o B$ if and only if $A \approx_{\mathit{s}} B$.
\end{proposition}
%\marginpar{L'unica cosa da provare \`e in realt\`a il Lemma 1}
\begin{proof}%[of Proposition~\ref{prop:weaksbequivobs}]
	The proof proceeds as follows.
	\begin{description}
		\item[From $\approx_{\mathit{s}}$ to $\sim_o$.] Assume  $\langle A, \bot \rangle \approx_{\mathit{s}} \langle B, \bot \rangle$ and take a $\otimes$-compact $c \in \mathcal{C}^\otimes$. Let
		
		\begin{equation}\label{comp:1}\langle A, c \rangle \longrightarrow \langle A_0, \sigma_0 \rangle \longrightarrow \langle A_1, \sigma_1 \rangle \longrightarrow \dots \longrightarrow \langle A_n, \sigma_n \rangle \dots \longrightarrow \dots
		\end{equation}
		\begin{equation}\label{comp:2}\langle B, c \rangle \longrightarrow \langle B_0, \rho_0 \rangle \longrightarrow \langle B_1, \rho_1 \rangle \longrightarrow \dots \longrightarrow \langle B_n, \rho_n \rangle \dots \longrightarrow \dots
		\end{equation}
		
		be two fair computations. Since $\approx_{\mathit{s}}$ is upward closed, 
		$\langle A, c \rangle \approx_{\mathit{s}} \langle B, c \rangle$ and thus $\langle B, c\rangle \Downarrow_{\sigma_i}$ for all $\sigma_i$. By Lemma~\ref{lem:barbsfair}, it follows that there exists an $\rho_j$ (in the above computation) such that 
		$\exists_{\Gamma_i} \sigma_i \leq \sigma_i \leq \exists_{\Gamma'_j} \rho_j$, and analogously for all $\rho_i$.
		%there exists a $\sigma_j$ such that $\rho_i \leq \sigma_j$. 
		Then $\sigma_0 \leq \sigma_1 \leq \dots$  and $\rho_0 \leq \rho_1 \leq \dots$ are cofinal and by Lemma~\ref{lem:cofinality}, it holds that $\bigvee_i \exists_{\Gamma_i} \sigma_i = \bigvee_i \exists_{\Gamma'_i} \rho_i$, which means 
		$\mathit{Result}(\langle A, c \rangle) = \mathit{Result}(\langle B, c \rangle)$.
		%\marginpar{before it was $\bigvee_i \Gamma_i \sigma_i$, no $\exists$}
		
		\item[From $\sim_o$ to $\approx_{\mathit{s}}$.] Assume $A \sim_o B$. First, we show that $\langle A, c\rangle$ and $\langle B, c\rangle$ satisfy the same weak barbs  for all $c \in \mathcal{C}$. Let (\ref{comp:1}) and (\ref{comp:2}) be two fair computations. Since $A \sim_o B$, then $\bigvee_i \exists_{\Gamma_i} \sigma_i = \bigvee_i \exists_{\Gamma'_j} \rho_i$. Since all (the projections of) the intermediate stores of the computations are $\otimes$-compact,
		%(by Lemma~\ref{comp}),
		%\marginpar{da rivedere la sintassi per la tell}
		then by Lemma~\ref{lem:cofinality}, for all $\sigma_i$ there exists an $\rho_j$ such that $\exists_{\Gamma_i} \sigma_i \leq \exists_{\Gamma'_j} \rho_j$. 
		Now suppose that $\langle A, c \rangle \Downarrow_d$. By Lemma~\ref{lem:barbsfair}, there exists a $\sigma_i$ 
		%(in the above computation) 
		such that $d \leq \exists_{\Gamma_i} \sigma_i$. Thus 
		%$d \leq \exists_{\Gamma_i} \sigma_i \leq \exists_{\Gamma_i} \rho_j$ that means 
		$\langle B, c\rangle \Downarrow_d$.
		
		It is now easy to prove that
		$R = \{(\gamma_1, \gamma_2) \mid \exists c. \langle A, c \rangle \longrightarrow^* \gamma_1 \& \langle B, c\rangle \longrightarrow^* \gamma_2\}$
		is a weak saturated bisimulation (Definition~\ref{def:weaksb}). Take $(\gamma_1 , \gamma_2 ) \in R$.
		If $\gamma_1 \Downarrow_d$ then $\langle A, c\rangle \Downarrow_d$ and, by the above observation, $\langle B, c\rangle \Downarrow_d$. Since \SCCP is
		confluent, also $\gamma_2 \Downarrow_d$.
		The fact that R is closed under $\longrightarrow^*$ is evident from the definition of $R$. While
		for proving that R is upward-closed take $\gamma_1 = \langle A', \sigma'\rangle$ and $\gamma_2 = \langle B', \rho'\rangle$. By Lemma~\ref{opmonotonicity}  for all $a \in \mathcal{C}, \langle A, c \otimes a\rangle \longrightarrow^* \langle A', \sigma' \otimes a \rangle$ and $\langle B, c \otimes a\rangle \longrightarrow^* \langle B', \rho'  \otimes a \rangle$. Thus, by definition of $R$, $(\langle A',\sigma' \otimes a \rangle, \langle B',\rho' \otimes a\rangle) \in R$. \qed
	\end{description} 
\end{proof}
\end{document}