\documentclass[main.tex]{subfiles}
\begin{document}

\section{A Labelled Transition System for Soft CCP}\label{sec:ltsSCCP}\label{sec:4}
Although $\approx_{\mathit{s}}$ is fully abstract, it is somewhat unsatisfactory
because of the upward-closure, i.e., the quantification in condition \emph{3} of Definition~\ref{def:weaksb}.

\begin{definition}[Labelled reductions]
	Let $\Gamma = {\mathcal A} \times \C^\otimes$ be the set of \emph{configurations} and $V$ the set of variables.
	The  \emph{labelled direct reduction semantics} for SCCP is the pair 
	$\langle \Gamma,  \mapsto \rangle$
	such that $\mapsto \, \, \subseteq \, \,\Gamma \times \Gamma$ is indexed over the couple
	$\langle \mathcal{C}^\otimes, 2^V \rangle$,
	%$2^V$,
	%between them, 
	i.e., $\mapsto = \bigcup_{\alpha \in \mathcal{C}^\otimes, \Delta \subseteq V} \xmapsto{\, \A \,}_\Delta$ and 
	$\xmapsto{\,\A\,}_\Delta \, \, \subseteq \, \,\Gamma \times \Gamma$, obtained by the rules in 
	Table~\ref{fig:ALTS}.
	
	The \emph{labelled reduction semantics} for SCCP is the pair 
	$\langle \Gamma,  \rightarrow \rangle$
	such that $\rarrow \, \, \subseteq \, \,\Gamma \times   \Gamma$ is the family 
	of binary relations indexed over the couple
	$\langle \mathcal{C}^\otimes, 2^V \rangle$;
	$\xrightarrow{\,\A\,}_\Delta \, \, \subseteq \, \,\Gamma \times \Gamma$ can obtained by the rules in 
	Table~\ref{fig:ALTS} and Table~\ref{fig:CRLTS}.
\end{definition}


In Table~\ref{fig:ALTS} and Table~\ref{fig:CRLTS} we refine the notion of transition (respectively given in Table~\ref{fig:operational} and Table~\ref{fig:operational2})
by adding a label that carries additional information about the constraints that cause the reduction.
Hence, we define a new labelled transition system (LTS) obtained by the family of relations
$\xmapsto{\;  \alpha \;}_\Delta \, \, \subseteq \, \,\Gamma \times \Gamma$ indexed over 
$\langle \mathcal{C}^\otimes, 2^V \rangle$;
as a reminder, $\Gamma$ is the set of configurations, $\mathcal{C}^\otimes$ the set of $\otimes$-compact constraints, and,  as for the unlabelled semantics in Section~\ref{sec:detSCCP}, 
transitions are indexed by sets of variables.
Rules in Table~\ref{fig:ALTS} and Table~\ref{fig:CRLTS} are identical to those in Table~\ref{fig:operational} and Table~\ref{fig:operational2}, except for a constraint $\alpha$ that
represents the minimal information that must be added to $\sigma$ in order to fire an action
from $\langle A, \sigma\rangle$  to $\langle A', \sigma' \rangle$, i.e., $\langle A, \sigma \otimes \alpha\rangle \longrightarrow_\Delta \langle A' , \sigma' \rangle$.


%%%%% OLD LTS

%\def\odivv{\; {\ominus\hspace{-4pt} \div} \;}
%\def\odivvv{\; {\ominus\hspace{-6pt} \div} \;}
%\begin{table}%[h]
%  %\hfill
%  \scalebox{0.87}{
%  \begin{minipage}{0.5\linewidth}%{.45\textwidth}
%    \begin{center}
%    \begin{tabular}{llll}
%    \mbox{\bf LR1}& $\langle \hbox{\tell}(c), \sigma \rangle \xrightarrow{\; \bot \;} \langle \mathit{\ostop},
%          \sigma \otimes c \rangle$
%    \ \ \ & \bf{Tell}&
%    \\
%   &\mbox{   }&\mbox{   } &\mbox{   }
%   \\
%   \mbox{\bf LR3}& $\frac {\displaystyle \langle A,\sigma \rangle \xrightarrow{\;  \alpha \;} \langle A', \sigma'   \rangle}
%              {\displaystyle
%               \langle A\parallel B, \sigma \rangle \xrightarrow{\;  \alpha \;} \langle A'\parallel B, \sigma'
%               \rangle}$& \bf{Par1}&
%  \\
%  & \mbox{   }&\mbox{   }&
%  \\
%  \mbox{\bf LR5}& $\frac {\displaystyle p(x) = A \in  \mathcal{P} }
%  {\displaystyle\langle p(x),\sigma\rangle \xrightarrow{\;  \bot \;} \langle  A, \sigma\rangle}$
%  &\bf{Rec1}&
%  \end{tabular}
%  \end{center}
% \end{minipage}
% %
% \hspace{0.6cm}
%  \begin{minipage}{0.5\linewidth}
%    \begin{center}
%    \begin{tabular}{llll}
%   \mbox{\bf LR2}& $\langle \hbox{\ask}(c) \rrarrow A, \sigma \rangle \xrightarrow{c  \odivv  \sigma}
%               \langle A, \sigma \otimes (c\odivvv \sigma) \rangle$
%   \ \ \ & \bf{Ask}&
%   \\
%   &\mbox{   }&\mbox{   }&\mbox{   }
%   \\
%   \mbox{\bf LR4}& $\frac {\displaystyle \; \; \; \langle A,\sigma\rangle \xrightarrow{\;  \alpha \;}
%               \langle A', \sigma'\rangle \; \; \;}
%               {\displaystyle \langle B\parallel A, \sigma \rangle \xrightarrow{\;  \alpha \;} \langle B\parallel A', \sigma' \rangle}$
%   & \bf{Par2}&
%   \\
%%&\mbox{   }&\mbox{   } &\mbox{   }
%%\\
%%\mbox{\bf LR5}& $\frac {\displaystyle \langle A[y/x],  \sigma
%%\rangle \xrightarrow{\;  \alpha \;} \langle B, \sigma' \rangle} {\displaystyle\langle \exists X.A,
%%\sigma \rangle \xrightarrow{\;  \alpha \;} \langle B, \sigma' \rangle} \ \  \text{TO DO}$
%%&\bf{Hide} &
%%\\
%   &\mbox{   }&\mbox{   }&
%  \\
%  \mbox{\bf LR6}& $\frac {\displaystyle x \neq y, \quad \displaystyle p(x) = A \in  \mathcal{P} }
%  {\displaystyle\langle p(y),\sigma\rangle \xrightarrow{\;  \bot \;} \langle \exists_x^{\delta_{x,y}} A, \sigma\rangle}$
%  &\bf{Rec2}&
%%\mbox{\bf LR6}& $\frac {\displaystyle \langle A[\hat{y}/\hat{x}],\sigma\rangle \xrightarrow{\;  \alpha \;} \langle
%%B, \sigma' \rangle} {\displaystyle\langle p(\hat{y}),\sigma\rangle \xrightarrow{\;  \alpha \;} \langle B,
%%\sigma'\rangle} \ \  {\it p(\hat{x}) :: A \in F}$
%%&\bf{Call}&\\
%%\mbox{\bf LR6}& $\frac {\displaystyle \langle A, \sigma \otimes d_{\hat{x}, \hat{y}}\rangle \xrightarrow{\;  \alpha \;} \langle
%%B, \sigma' \rangle} {\displaystyle\langle p(\hat{y}),\sigma\rangle \xrightarrow{\;  \alpha \;} \langle B,
%%\sigma'\rangle} \ \  {\it p(\hat{x}) :: A \in F}$
%%&\bf{P-call}&\\
%  \end{tabular}
%  \end{center}
% \end{minipage}
%}
%
%\def\odiv{\, {\ominus\hspace{-6pt} \div} \,}
%\begin{center}
%\scalebox{0.9}{
%\begin{tabular}{llll}
%\mbox{\bf LR7}&
%%$\frac {\displaystyle \la A[z/x], c[z/x] \otimes \sigma \ra \xrightarrow{\;  \alpha \;} \la
%%B,  c' \otimes \sigma \otimes \alpha \ra} {\displaystyle\la \exists^c_x A,
%%\sigma\ra \xrightarrow{\;  \alpha \;} \la \exists^{c'[x/z]}_x B[x/z], \exists_x(c'[x/z]) \otimes \sigma \otimes \alpha \ra} \ \
%% {\it x\not\in fv(c'), z \not\in (fv(A) \cup fv(c \otimes \sigma \otimes \alpha)) }$
%$\frac {\displaystyle \la A, \sigma' \otimes \sigma_0[^z/_x] \ra \xrightarrow{\;  \alpha \;}
%            \la B,  \sigma" \ra\ \ \ \mbox{  \textit{with}  } \sigma_0 = \sigma \odiv \exists_x \sigma'}
%           {\displaystyle\la \exists^{\sigma'}_x A, \sigma\ra \xrightarrow{\;  \alpha[^x/_z] \;}
%            \la \exists^{\sigma_1}_x B, \sigma_0 \otimes \alpha[^x/_z] \otimes \exists_x\sigma_1 \ra
%            \ \ \ \mbox{  \textit{with}  } \sigma_1 = \sigma" \odiv (\alpha \otimes \sigma_0[^z/_x])}$ \ \ , * \; \
%&\bf{Hide}&\\
%   &\mbox{   }&\mbox{   }&
%  \\
%&\footnotesize$* \equiv {\it x\not\in sv(\alpha), z \not\in fv(A) \cup sv(\sigma)\cup sv(\sigma') }$&
%\end{tabular}
%}
%\end{center}
%%
%\caption{An LTS for \SCCP.}
%\label{fig:LTS}
%\end{table}
%\def\odiv{\, {\ominus\hspace{-6.8pt} \div} \,}
%\def\odivvv{\; {\ominus\hspace{-4.7pt} \div} \;}



%%%%% NEW LTS
\def\odiv{\; {\ominus\hspace{-6.8pt} \div} \;}
\def\odivv{\; {\ominus\hspace{-5.2pt} \div} \;}
\def\odivvv{\; {\ominus\hspace{-7.8pt} \div} \;}
\begin{table}  %\hfil5

   \begin{center}
   	  \scalebox{0.9}{
   \begin{tabular}{llll} 
   %
   \mbox{\bf LA1}& $\frac{\displaystyle  sv(\sigma) \cup sv(c) \subseteq \Delta } {\displaystyle \langle \hbox{\tell}(c), \sigma \rangle \xmapsto{\;  \bot \;}_\Delta  \langle \hbox{\ostop},
                                               \sigma \otimes c\rangle}$
   \ \ \ & \bf{Tell}&
  \\ 
  &\mbox{   }&\mbox{   } &\mbox{   }
  \\
  \mbox{\bf LA2}& $\frac {\displaystyle   sv(\sigma) \cup sv(c)  \cup fv(A) \subseteq \Delta}{\displaystyle
  	\begin{array}{l} \langle \hbox{\ask}(c) \rightarrow A, \sigma \rangle \xmapsto{c  \odivv  \sigma}_\Delta
  	\langle A, \sigma \otimes (c\odivvv \sigma) \rangle
  	\end{array}}$
  \ \ \ & \bf{Ask}&
  \\
  &\mbox{   }&\mbox{   }&
  \\
  \mbox{\bf LA3}& $\frac {\displaystyle sv(\sigma) \cup  \{y\} \subseteq \Delta \wedge \displaystyle p(x) = A \in  \mathcal{P} }
  {\displaystyle\langle p(y),\sigma\rangle \xmapsto{\bot}_\Delta \langle  A[^y/_x], \sigma\rangle}$ 
  &\bf{Rec}&
    \\ \\
\mbox{\bf LA4}& $\frac {\displaystyle sv(\sigma) \cup fv(\exists_x A) \subseteq \Delta \wedge w \not \in \Delta }
{\displaystyle\langle \exists_x A,\sigma\rangle \xmapsto{\bot}_\Delta \langle A[^w/_x], \sigma\rangle}$
&\bf{Hide}&

  \end{tabular}
}
  \end{center}

\caption{Axioms of the labelled semantics for \SCCP.}
\label{fig:ALTS}
\end{table}
\def\odiv{\, {\ominus\hspace{-7.8pt} \div} \,}
\def\odivvv{\; {\ominus\hspace{-4.7pt} \div} \;}


\def\odiv{\; {\ominus\hspace{-6.8pt} \div} \;}
\def\odivv{\; {\ominus\hspace{-5.2pt} \div} \;}
\def\odivvv{\; {\ominus\hspace{-7.8pt} \div} \;}
\begin{table}  %\hfil5

   \begin{center}
   	  \scalebox{0.9}{
   \begin{tabular}{llll} 
   %
  \mbox{\bf LR1}& $\frac {\displaystyle \langle A,\sigma \rangle \xrightarrow{\alpha}_\Delta \langle A', \sigma' \rangle
  \wedge fv(B) \subseteq \Delta} 
  {\displaystyle \begin{array}{l}
                          \langle A\parallel B, \sigma \rangle \xrightarrow{\alpha}_\Delta \langle A'\parallel B, \sigma' \rangle
                          \end{array}}$ 
    & \bf{Par1}&
  \\
  & \mbox{   }&\mbox{   }&
  \\
    \mbox{\bf LR2}& $\frac {\displaystyle \langle A,\sigma \rangle \xrightarrow{\alpha}_\Delta \langle A', \sigma'   \rangle
    	\wedge fv(B) \subseteq \Delta} 
    {\displaystyle 
    	\begin{array}{l} \langle B\parallel A, \sigma \rangle \xrightarrow{\alpha}_\Delta \langle B\parallel A', \sigma' \rangle
    	\end{array}}$& \bf{Par2}&
    \\
    &\mbox{   }&\mbox{   }&
  \end{tabular}
}
  \end{center}

\caption{Contextual rules of the labelled semantics for \SCCP.}
\label{fig:CRLTS}
\end{table}
\def\odiv{\, {\ominus\hspace{-7.8pt} \div} \,}
\def\odivvv{\; {\ominus\hspace{-4.7pt} \div} \;}


Rule {\bf LA2} says that $\langle \mathit{\ask}(c) \rightarrow A, \sigma \rangle$
can evolve to $\langle A, \sigma \otimes \alpha \rangle$ if the environment provides a minimal
constraint $\alpha$ that added to the store $\sigma$ entails $c$, i.e., $\alpha = c \odiv \sigma$.
Notice that, differently from \cite{pippo}, here the definition of this minimal label comes directly
from a derived operator of the underlying CLIM (i.e., from $\odiv$), which by Lemma~\ref{preres} preserves $\otimes$-compactness.
%
%In rule {\bf LR7}  we rename the global $x$ with a fresh variable $z$, instead of hiding $x$ in the
%global store, as we do in the corresponding unlabelled rule, i.e. {\bf R7}. We accomplish this in order
%to keep track of the global $x$ in $\alpha$: finally, in the final
% result we restore all the occurrences of $z$ back to $x$, i.e., $\alpha[x/z]$.
%\marginpar{discuss side conditions}
%
%To better explain {\bf LR7} and {\bf R7} consider the following example.
%
%\begin{example}
%Let $c_x$ and $c_y$ constraints with $fv(c_x)=x$, $fv(c_y)=y$, $c_x \not\leq c_y$ and $c_y \not\leq c_x$.
%Consider the configuration $\config{\localp{x}{\askp{(c_y)}{\tellp{c_x}}}{\mbox{{\scriptsize \true}}}}{c_x}$.
%In order to proceed with the local transition we first need to make sure that the global $x$ will not
%clash with the local $x$, to achieve this we shall rename the $x$ in the global store $c_x$ with a fresh variable $z$,
%namely $c_x\rename{z}{x} = c_z = \Si_0\rename{z}{x}$.
%Therefore the local transition looks like this $\transition{\askp{(c_y)}{\tellp{c_x}}}{c_z}{c_y}{\tellp{c_x}}{c_z \otimes c_y}$.
%Using {\bf LR7} we have
%$\config{\localp{x}{\askp{(c_y)}{\tellp{c_x}}}{\mbox{{\scriptsize \true}}}}{c_x} \trans{c_y}
%\config{\localp{x}{\tellp{c_x}}{\mbox{{\scriptsize \true}}}}{c_x \otimes c_y}$.
%Now the new local transition is  $\transition{\tellp{c_x}}{c_z \otimes c_y}{}{\Stop}{c_z \otimes c_y \otimes c_x}$
%since $\Si_0\rename{z}{x} = ((c_x \otimes c_y) \odiv \exists_x \truep)\rename{z}{x} = (c_x \otimes c_y) \rename{z}{x} = c_z \otimes c_y$.
%Therefore the final transition is $\config{\localp{x}{\tellp{c_x}}{\mbox{{\scriptsize \true}}}}{c_x \otimes c_y} \trans{}
%\config{\localp{x}{\Stop}{c_x}}{c_x \otimes c_y \otimes \exists_x c_x}$ and indeed $\Si_1 = (c_z \otimes c_y \otimes c_x) \odiv (c_z \otimes c_y) = c_x$
%captures exactly the information produced locally.
%Note that if we compute $\Si_0$ in the last configuration, it corresponds exactly to the global store minus
%the existential quantification of the local store.
%Thus avoiding several copies of the local information to be used in the premise of {\bf LR7} and {\bf R7}.
%More concretely, $\Si_0 = (c_x \otimes c_y \otimes \exists_x c_x) \odiv (\exists_x c_x) = c_x \otimes c_y$.
%\end{example}


The LTS is sound and complete with respect to the unlabelled semantics. 
%Soundness states
%that $\langle A, \sigma\rangle \trans{\A}_\Delta \langle A' , \sigma' \rangle$ implies
%that if $\alpha$ is added to $\sigma$, $A$ can reach $\langle A', \sigma'\rangle$. Completeness states that
%if we add $c$ to (the store in) $\langle A, \sigma \rangle$ and reduce to $\langle A', \sigma' \rangle$,
%there is $\alpha \leq c $ such that $\langle A, \sigma \rangle \xrightarrow{\;  \alpha \: }_\Delta \langle A', \sigma'' \rangle$ with
%$\sigma'' \leq \sigma'$.
%\marginpar{Check soundness and completeness}

%For technical purposes, in the following lemmata we shall use an equivalent formulation of \textbf{R7}
%that uses renaming instead of hiding using the existential quantification, as follows:
%{\footnotesize
%\[
%\makebox{\textbf{R7'} }
%\bigfrac{\config{B}{\Si' \otimes \Si_0\rename{z}{x}} \trans{} \pairccp{B'}{\Si''} \mbox{ with } \Si_0 = (\Si \odiv \exists_x \Si')}
%{\config{\localp{x}{B}{\Si'}}{\Si} \trans{} \config{\localp{x}{B'}{\Si_1}}{\Si_0 \otimes \exists_x \Si_1} \mbox{ with } \Si_1 = (\Si'' \odiv \Si_0\rename{z}{x})}
%\mbox{ where } z \not\in fv(A) \cup sv(\Si) \cup sv(\Si')
%\]
%}
%One can easily verify that \textbf{R7} and \textbf{R7'} coincide.

\begin{lemma}[Soundness]\label{lemma:soundness}
% If $\langle A, \sigma\rangle \xrightarrow{\;  \alpha \: } \langle A', \sigma' \rangle$ then $\langle A, \sigma \otimes \alpha \rangle \longrightarrow \langle A', \sigma' \rangle$.
Let $A$ be an agent and $\sigma \in \mathcal{C}^\otimes$.
%
If $\config{A}{\Si} \xmapsto{\,\A\,}_\Delta \config{A'}{\sigma'}$ then $\config{A}{\Si \otimes \A}  \mapsto_\Delta \config{A'}{\sigma'}$.
%%% Proof by Luis
\end{lemma}

\begin{proof}%[of Lemma~\ref{lemma:soundness}]
	
%We proceed by induction on (the depth) of the inference of 
%$\config{A}{\Si} \trans{\A}_\Delta \config{A'}{\rho}$.
	We just consider {\bf LA2}: The other cases are straightforward.
	% \begin{itemize}
	%  \item Using \textbf{LR2} then $A = (\askp{(c)}{A'}$), $\A = c \odiv \Si$ and $\rho = (\Si \otimes (c \odiv \Si)) = (\Si \otimes \A)$.
	% Thus we know that $c \leq (\Si \otimes (c \odiv \Si))$ then by using \textbf{R2}
	% $\config{A}{\Si \otimes \A} \trans{} \config{A'}{\rho}$.
	%  \item Using \textbf{LR4} then $A = \localp{x}{B}{\Si'}$, $\A = \A'\rename{x}{z}$ and $A' = \localp{x}{B'}{\Si_1}$ thus the transition
	% is $\config{\localp{x}{B}{\Si'}}{\Si} \trans{\A'\rename{x}{z}} \config{\localp{x}{B'}{\Si_1}}{\rho}$.
	% By appeal to induction we have
	% \end{itemize}
	
	By \textbf{LA2} we have $A = \askp{c}{A'}$, $\A = c \odiv \Si$, 
	and $\sigma' = (\Si \otimes (c \odiv \Si))$.
	Since it always holds that $c \leq (\Si \otimes (c \odiv \Si))$ then by \textbf{A2}
	we have $\config{A}{\Si \otimes \A} \mapsto_\Delta \config{A'}{\sigma'}$.
	%
	%Using \textbf{LR7} then $A = \localp{x}{B}{\Si'}$, $\A = \A'\rename{x}{z}$ and $A' = \localp{x}{B'}{\Si_1}$ thus the transition
	%is $\config{\localp{x}{B}{\Si'}}{\Si} \trans{\A'\rename{x}{z}} \config{\localp{x}{B'}{\Si_1}}{\rho}$.
	%By appeal to induction we have $\config{B}{\Si' \otimes \Si_0\rename{z}{x}} \trans{\A'} \pairccp{B'}{\Si''}$
	%where $\Si_0 = (\Si \odiv \exists_x \Si')$ and $\Si_1 = (\Si'' \odiv (\A' \otimes \Si_0\rename{z}{x}))$.
	%Then $\rho = (\Si_0 \otimes \A'\rename{x}{z} \otimes \exists_x \Si_1)$ and $z \not\in fv(A) \cup sv(\Si) \cup sv(\Si')$ and $x \not\in sv(\A)$.
	%By induction hypothesis we obtain $\config{B}{\Si' \otimes \Si_0\rename{z}{x} \otimes \A'} \trans{} \pairccp{B'}{\Si''}$.
	%Now notice that $\A = \A'\rename{x}{z}$ then $\A\rename{z}{x} = \A'$.
	%By replacing $\A'$ in the previous transition $\config{B}{\Si' \otimes \Si_0\rename{z}{x} \otimes \A\rename{z}{x}} \trans{} \pairccp{B'}{\Si''}$
	%which is equivalent to $\config{B}{\Si' \otimes (\Si_0 \otimes \A)\rename{z}{x}} \trans{} \pairccp{B'}{\Si''}$ {\bf (1)}.
	%Let $\Si'_0 = (\Si_0 \otimes \A)$, now consider $\Si_0 = (\Si \odiv \exists_x \Si')$ then
	%$\Si_0 \otimes \exists_x \Si'= (\Si \odiv \exists_x \Si') \otimes \exists_x \Si'$ and since $\exists_x \Si' \leq \Si$ then by invertibility
	%$\Si_0 \otimes \exists_x \Si'= \Si$. From the previous equation we can see that $\Si \otimes \A = \Si_0 \otimes \exists_x \Si' \otimes \A$,
	%moreover $((\Si \otimes \A) \odiv  \exists_x \Si') = ((\Si_0 \otimes \exists_x \Si' \otimes \A) \odiv \exists_x \Si') \leq (\Si_0 \otimes \A) = \Si'_0$
	%therefore $((\Si \otimes \A) \odiv  \exists_x \Si') \leq \Si'_0$.
	%Similarly $\Si'_0 = (\Si_0 \otimes \A) = ((\Si \odiv  \exists_x \Si') \otimes \A) \leq
	%((\Si \otimes \A) \odiv  \exists_x \Si')$\footnote{$b \otimes (x \odiv a) \leq (b \otimes x) \odiv a$, see (f.12) in \cite[Table~4.1]{resbook}, the ORDER IS INVERTED!!}
	%hence $\Si'_0 \leq ((\Si \otimes \A) \odiv  \exists_x \Si')$ thus $\Si'_0 = (\Si \otimes \A) \odiv \exists_x \Si'$.
	%Using $\Si'_0$ then by R7' and {\bf (1)} we get
	%$\config{\localp{x}{B}{\Si'}}{\Si \otimes \A} \trans{} \config{\localp{x}{B'}{\Si'_1}}{\Si'_0 \otimes \exists_x \Si'_1}$.
	%To finish the proof we need to show that $\Si'_1 = \Si_1$,
	%note that $\Si'_1 = (\Si'' \odiv \Si'_0\rename{x}{z}) = (\Si'' \odiv ((\Si_0 \otimes \A) \rename{x}{z})) =
	%(\Si'' \odiv (\A' \otimes \Si_0\rename{x}{z})) = \Si_1$.
	%Finally $\config{\localp{x}{B}{\Si'}}{\Si \otimes \A} \trans{} \config{\localp{x}{B'}{\Si_1}}{\rho}$
\end{proof}

We are going to need a stronger notion of completeness. Intuitively, whenever there exists a direct reduction $\langle A, \sigma \rangle \mapsto$, then a corresponding labelled direct reduction exists originating from any $\langle A, \rho \rangle$ such that $\rho \leq \sigma$.

\begin{lemma}[Completeness]\label{lemma:completeness}
% If $\langle A, \sigma \otimes d \rangle \longrightarrow \langle A', \sigma' \rangle$ then there exist $\alpha, a \in C^\otimes$ such that $\langle A, \sigma\rangle \xrightarrow{\;  \alpha \: } \langle A', \sigma''\rangle$ and $\alpha \otimes a = d$, $\sigma'' \otimes a = \sigma'$.
Let $A$ be an agent and $\sigma, \rho \in \mathcal{C}^\otimes$ such that $\rho \leq \sigma$.
%
Moreover, let us assume $\mathcal{C}$ to be invertible.
%
If $\config{A}{\Si} \mapsto_\Delta \config{A'}{\sigma'}$ then there exist $\A, a \in C^\otimes$ such that
$\config{A}{\rho} \xmapsto{\, \A \,}_\Delta \config{A'}{\rho'}$ with $\rho \otimes \A \otimes a = \sigma$ and $\rho' \otimes a = \sigma'$.
\end{lemma}

\begin{proof}%[of Lemma~\ref{lemma:completeness}]
	%We proceed by induction on (the depth) of the inference of 
	%$\config{A}{\Si \otimes d} \trans{}_\Delta \config{A'}{\rho}$.
	We just consider {\bf LA2}: The other cases are straightforward
	and they are verified by always choosing $\alpha = a = \bot$.
	
	By \textbf{LA2} we have $A = \askp{c}{A'}$, $\sigma' = \Si$, and $c \leq \Si$.
	Now consider $\config{A}{\rho} \xmapsto{\,\A\,}_\Delta \config{A'}{\rho'}$, where 
	$\A = c \odiv \rho$ and $\rho' = \rho \otimes \A$.
	Also, let us note that $c \leq \sigma$ and $\rho \leq \sigma$ imply $\rho' \leq \sigma$
	by invertibility.
	Thus, we can take $a = \sigma \odiv \rho'$, and the conditions 
	are easily verified again by the invertibility of $\mathcal{C}$.
%	
%	$\A \otimes a = (c \odiv \Si) \otimes (d \odiv (c \odiv \Si)) = d$
%	and finally $\rho' \otimes a = \Si \otimes \A \otimes a = \Si \otimes d = \rho$.
	%
	%Using \textbf{LR7} then $A = \localp{x}{B}{\Si'}$ and $A' = \localp{x}{B'}{\Si_1}$ thus the transition
	%we consider is of the form $\config{\localp{x}{B}{\Si'}}{\Si \otimes d} \trans{} \config{\localp{x}{B'}{\Si'_1}}{\rho}$.
	%By appeal to induction $\config{B}{\Si' \otimes \Si'_0\rename{z}{x}} \trans{} \pairccp{B'}{\Si''}$
	%where $\Si'_0 = ((\Si \otimes d) \odiv \exists_x \Si')$ and $\Si'_1 = (\Si'' \odiv \Si'_0\rename{z}{x})$,
	%hence $\rho = \Si'_0 \otimes (\exists_x \Si'_1)$.
	%Also $z \not\in fv(B) \cup sv(\Si) \cup sv(d) \cup sv(\Si')$.
	%Furthermore, it is safe to assume that $\Si'' = \Si' \otimes \Si'_0\rename{z}{x} \otimes \B$
	%where $\B$ is a constraint that may have been produced by a tell process in $B$.
	%Let $\Si_0 = \Si \odiv \exists_x \Si'$ and by invertibility $\Si = \Si_0 \otimes \exists_x \Si'$,
	%hence note that $\Si'_0 = ((\Si \otimes d) \odiv \exists_x \Si') = ((\Si_0 \otimes (\exists_x \Si') \otimes d) \odiv \exists_x \Si') = \Si_0 \otimes d$.
	%Thus we have the transition $\config{B}{\Si' \otimes (\Si_0 \otimes d)\rename{z}{x}} \trans{} \pairccp{B'}{\Si''}$
	%which is equivalent to $\config{B}{\Si' \otimes \Si_0\rename{z}{x} \otimes d\rename{z}{x}} \trans{} \pairccp{B'}{\Si''}$.
	%By induction hypothesis there exists $\A'$ and $a'$ such that $\config{B}{\Si' \otimes \Si_0\rename{z}{x}} \trans{\A'} \pairccp{B'}{\Si'''}$ {\bf (1)}
	%where $\A' \otimes a' = d\rename{z}{x}$ and $\Si''' \otimes a' = \Si''$.
	%Note that $x \not\in fv(d\rename{z}{x})$ therefore $x \not\in fv(\A' \otimes a')$, namely $x \not\in fv(\A') \cup fv(a')$.
	%Also we have that $\Si''' = \Si' \otimes \Si_0\rename{z}{x} \otimes \A' \otimes \B$ where
	%$\B$ is a constraint that may have been produced by a tell process in $B$.
	%By \textbf{LR7} and {\bf (1)} we obtain $\transition{\localp{x}{B}{\Si'}}{\Si}{\A'\rename{x}{z}}{\localp{x}{B'}{\Si_1}}{\rho'}$
	%where $\rho' = (\Si_0 \otimes \A'\rename{x}{z} \otimes \exists_x \Si_1)$ and $\Si_1 = \Si''' \odiv (\A' \otimes \Si_0\rename{z}{x})$.
	%In order to conclude we first need to prove that $\Si_1 = \Si'_1$.
	%By definition $\Si'_1 = (\Si'' \odiv \Si'_0\rename{z}{x})$
	%then $\Si'_1 = (\Si' \otimes \Si'_0\rename{z}{x} \otimes \B) \odiv \Si'_0\rename{z}{x})$ thus $\Si'_1 = \Si' \otimes \B$.
	%Similarly $\Si_1 = \Si''' \odiv (\A' \otimes \Si_0\rename{z}{x})$ and by definition of $\Si'''$ then
	%$\Si_1 = (\Si' \otimes \Si_0\rename{z}{x} \otimes \A' \otimes \B) \odiv (\A' \otimes \Si_0\rename{z}{x}) = \Si' \otimes \B$,
	%therefore $\Si_1 = \Si'_1$.
	%Finally take $\A = \A'\rename{x}{z}$, $a = a'\rename{x}{z}$ and let us verify
	%that $\A \otimes a = d$ and $\rho' \otimes a = \rho$ hold.
	%First $\A \otimes a = \A'\rename{x}{z} \otimes a'\rename{x}{z} = (\A' \otimes a')\rename{x}{z} = d\rename{z}{x}\rename{x}{z} = d$ and to conclude
	%$\rho' \otimes a = \Si_0 \otimes \A'\rename{x}{z} \otimes (\exists_x \Si_1) \otimes a'\rename{x}{z} = \Si_0 \otimes d \otimes (\exists_x \Si_1) = \Si'_0 \otimes (\exists_x \Si_1) = \Si'_0 \otimes (\exists_x \Si'_1) =\rho$.
\end{proof}

The complex statement boils down to the checking for the satisfaction of a constraint $c$
 in a store $\rho$ that does not necessarily verify it.
%
This is the equivalent of \cite[Lemma 5]{pippo} for the crisp language, and it is going to be needed in 
Lemma \ref{lemma:equality} of the next section to prove that weak and strong bisimilarities are 
upward closed with respect to the store.
\\

%\textcolor{red}{UN ESEMPIO SU COME MAI CI SERVE a? ANCHE COPIATO DAL LAVORO DI PIPPO}.\marginpar{???}
%\\

The lemmata above can in any case be simplified to state that the labelled reduction semantics is in fact an extension of the unlabelled one.

% \begin{proof}
% (TO FINISH)
% The proof proceeds by induction on (the depth) of the inference of $\la A, \sigma \otimes a\ra \longrightarrow  \la A' , \sigma'\ra$, and a case analysis on the last transition rule used. As before, the only relevant cases are  {\bf R2} and  {\bf R7}.
%
% As for {\bf R2},  by construction $\alpha = c \odiv \sigma$ and $\sigma'' = \sigma \otimes (c \odiv \sigma)$,
% thus it suffices to consider $a = d \odiv (c \odiv \sigma)$.
% Since the verification of the ask guarantees that $c \leq (\sigma \otimes d)$, we have that $(c \odiv \sigma) \leq d$ and by invertibility $\alpha \otimes a = d$ and $\sigma'' \otimes a = \sigma \otimes d = \sigma'$, hence the result holds.
%
% As for  {\bf R7}', $D= \exists^{\sigma'}_x A$ and $E= \exists^{\sigma_1}_x B$, and $\bar{\sigma} =  \sigma_0 \otimes \exists_x \sigma_1 \otimes d$,
% with $\la A, \sigma' \otimes  (\sigma_0 \otimes \alpha)[z/x] \ra \rarrow \la B,  \sigma'' \otimes \alpha[z/x] \ra$
% where $z \not\in (fv(A) \cup fv(\sigma') \cup fv(\sigma) \cup fv(\alpha))$, by a previous step of inference. By induction,
% we know that there exists $\alpha$ and $a$
% such that $\langle B, \sigma' \otimes \sigma[z/x] \rangle \xrightarrow{\;  \alpha \: } \langle B, \sigma'' \otimes \alpha \rangle$, with $ \alpha \otimes a = d[z/x]$, and $\sigma'' \otimes \alpha[z/x] = \sigma_1'' \otimes a$. Note that $x \not\in fv(d[z/x]) = fv(\alpha \otimes a)$,
% and thus $x \not\in (fv(\alpha) \cup fv(a))$. By using rule {\bf LR7}
% we have $\langle \exists_{x}^{\sigma'} A, \sigma\ra\xrightarrow{\;  \alpha[x/z] \;} \langle \exists^{\sigma_1}_xB, \sigma_0 \otimes \exists_x \sigma_1 \otimes \alpha[x/z] \rangle$. From $d[z/x] = \alpha \otimes a$
% we have $(d[z/x])[x/z] = (\alpha \otimes a)[x/z]$, that is $d= \alpha[x/z] \otimes a[x/z]$.
% Now, consider $\sigma_1'' = \sigma_0 \otimes \exists_{x}^{\sigma_1} \otimes \alpha[x/z]$;
% we have that $\sigma'' \otimes \alpha \otimes a[x/z] = \sigma_0 \otimes \exists_{x}^{\sigma_1} \otimes \alpha[x/z] \otimes a[x/z]$
% that, by the previous equivalence, is equal to $\sigma_0 \otimes \exists_{x}^{\sigma_1} \otimes d$, that is, $\bar{\sigma}$.
% \end{proof}

\begin{proposition}\label{cor:true}
Let $A$ be an agent and $\sigma \in \mathcal{C}^\otimes$.
Then $\langle A, \sigma \rangle \xrightarrow{\;  \bot \: }_\Delta \langle A', \sigma' \rangle$
if and only if $\langle A, \sigma\rangle \longrightarrow_\Delta \langle A', \sigma'\rangle$.
\end{proposition}

\subsection{Strong and Weak Bisimilarity on the LTS.}
%Having an LTS for SCCP, w
We now define an equivalence that
characterises  $\sim_{\mathit{s}}$ without the upward closure condition. 
As it occurs with the crisp language, and differently from calculi such as Milner's \CCS, 
%when using an LTS 
barbs cannot be removed
from the definition of bisimilarity because they cannot be inferred by the transitions.
%Equivalently, as proposed in ,
%an alternative (yet cumbersome) solution might have been to have labels showing the whole store.
%but this would have betrayed the philosophy of ``labels as minimal constraints''.


\begin{definition}[Strong bisimilarity]\label{def:strong} 
A strong bisimulation is a symmetric relation $R$
on configurations such that if $(\gamma_1, \gamma_2) \in R$ for $\gamma_1 = \langle A, \sigma \rangle$ and $\gamma_2 = \langle B, \rho\rangle$
\begin{enumerate}
\item if $\gamma_1 \downarrow_c$ then $\gamma_2 \downarrow_c$
\item if $\gamma_1 \xrightarrow{\; \; \alpha\;  \;} \gamma_1'$ then $\exists \gamma_2'$
such that $\langle B, \rho \otimes \alpha \rangle \longrightarrow \gamma_2'$ and $(\gamma_1', \gamma_2') \in R$.
\end{enumerate}
We say that $\gamma_1$ and $\gamma_2$ are strongly bisimilar ($\gamma_1 \sim \gamma_2$) if there exists a strong
bisimulation $R$ such that $(\gamma_1, \gamma_2) \in R$.
\end{definition}

Whenever $\sigma$ and $\rho$ are $\otimes$-compact elements, the first condition is equivalent to require $\sigma \leq \rho.$ 
%
Thus $(\gamma_1, \gamma_2) \in R$ would
 imply that $\gamma_1$ and $\gamma_2$ have the same store.
 %
 As for the second condition, we adopted a \emph{semi-saturated} equivalence, introduced 
 for CCP in~\cite{pippo}. In the bisimulation game a label can  be simulated 
 by a reduction including in the store the label itself.
 %, with the objective to obtain an equivalence which is a congruence.

\begin{definition}[Weak bisimilarity]\label{def:weak} A weak bisimulation is a symmetric
relation $R$ on configurations such that if $(\gamma_1, \gamma_2) \in R$ for $\gamma_1 = \langle A, \sigma \rangle$ and $\gamma_2 = \langle B, \rho\rangle$
\begin{enumerate}
\item if $\gamma_1 \downarrow_c$ then $\gamma_2 \Downarrow_c$
\item if $\gamma_1 \xrightarrow{\; \; \alpha\;  \;} \gamma_1'$ then $\exists \gamma_2'$ such that $\langle B, \rho \otimes \alpha \rangle \longrightarrow^* \gamma_2'$ and $(\gamma_1', \gamma_2') \in R$.
\end{enumerate}
We say that $\gamma_1$ and $\gamma_2$ are weakly bisimilar ($\gamma_1 \approx \gamma_2$) if there exists a weak bisimulation $R$ such that $(\gamma_1, \gamma_2) \in R$.
\end{definition}

With respect to the weak equivalence for crisp constraints, some of its characteristic equalities do not hold, e.g.
$\askp{c}{\tell({c})} \not  \approx \ostop$. As usual, this is due to the fact that the underlying CLIM may not be idempotent.

We can now conclude by proving the equivalence between $\sim_{\mathit{s}}$ and $\sim$ and between $\approx_{\mathit{s}}$ and $\approx$ (hence, $\approx$ is further equivalent to $\sim_o$, using Proposition~\ref{prop:weaksbequivobs}). We start by showing that $\sim$ is preserved under closure.

\begin{lemma}\label{lemma:equality}
Let $\langle A, \sigma\rangle$, $\langle B, \rho\rangle$ be configurations and $c \in C^\otimes$
 Moreover, let $\mathcal{C}$ be invertible. 
If $\langle A, \sigma\rangle \sim \langle B, \rho\rangle$, then $\langle A, \sigma \otimes c \rangle \sim \langle B, \rho \otimes c\rangle$.
\end{lemma}


\begin{proof}%[of Lemma~\ref{lemma:equality}]
	We need to show that $R = \{ (\langle A, \sigma \otimes a \rangle \sim \langle B, \rho \otimes a\rangle) \mid  \langle A, \sigma\rangle \sim \langle B, \rho\rangle\}$ satisfies the two properties in Definition~\ref{def:strong}.
	\begin{enumerate}[i)]
		\item From the hypothesis $\langle A, \sigma\rangle \sim \langle B, \rho\rangle$, we have that $\rho = \sigma$,
		%
		%, since $\langle A, \sigma\rangle \downarrow_\sigma$, then $\langle B, \rho\rangle \downarrow_\sigma$, that is $\rho
		%\geq \sigma$. For the same reason $\sigma \geq \rho$, and then $\sigma = \rho$,
		thus $\langle A, \sigma \otimes a \rangle$ and $\langle B, \rho \otimes a\rangle$ satisfy the same barbs.
		
		\item Let us assume $\langle A, \sigma \otimes c \rangle  \xrightarrow{\; \; \alpha\;  \;} \langle A', \sigma'\rangle$,
		we need to prove the existence of $B'$ and $\rho'$ such that $\langle B, \rho \otimes c \otimes \alpha \rangle  \rightarrow \langle B', \rho'\rangle$ and $(\langle A', \sigma' \rangle, \langle B', \rho'\rangle) \in R$. %\marginpar{da riaggiustare}
		By Lemma~\ref{lemma:soundness} and Lemma~\ref{lemma:completeness} we obtain $\langle A, \sigma \rangle \xrightarrow{\; \; \alpha'\;  \;} \langle A', \sigma''\rangle$ and the existence of $a'$ such that $\sigma \otimes \alpha' \otimes a' = \sigma \otimes c \otimes \alpha$ (1) and $\sigma'' \otimes a' = \sigma'$ (2).
		By invertibility, we also get that (1') $\alpha' \otimes a' = c \otimes \alpha$.
		From the labelled transition of $\langle A, \sigma\rangle$ and the hypothesis $\langle A, \sigma \rangle \sim \langle B, \rho \rangle$, we have that $\langle B, \rho \otimes \alpha' \rangle \rightarrow \langle B', \rho''\rangle$, with $\langle A, \sigma''\rangle \sim \langle B, \rho''\rangle$ (3). By (1') we have $\langle B, \rho \otimes c \otimes \alpha \rangle = \langle B, \rho \otimes \alpha' \otimes
		a' \rangle$ and $\langle B, \rho \otimes \alpha' \otimes a' \rangle \rightarrow \langle  B, \rho'' \otimes a' \rangle$ (due to
		operational monotonicity). Finally, by the definition of $R$ and (3), we conclude that $(\langle A', \sigma'' \otimes a' \rangle, \langle B', \rho'' \otimes a'\rangle) \in R$, and, by (2), $\langle A', \sigma'' \otimes a' \rangle = \langle A', \sigma'\rangle$.
	\end{enumerate} 
\end{proof}




\begin{theorem}\label{stronEq}
$\sim_{\mathit{s}} \; \subseteq \; \sim  \; \subseteq \;$. Moreover, let $\mathcal{C}$ be invertible.
Then $\sim_{\mathit{s}} \; = \; \sim$
\end{theorem}

\begin{proof}%[of Theorem~\ref{stronEq}]
	The inclusion $\sim \subseteq \sim_{s}$ can be proved by using Lemma~\ref{lemma:equality}.
	%
	\begin{description}
		\item[From $\sim$ to $\sim_{s}$.]
		We show that $R = \{ (\langle A, \sigma \rangle, \langle B, \rho \rangle) \mid  \langle A, \sigma\rangle \sim \langle B, \rho\rangle\}$
		is a saturated bisimulation, i.e.,
		if $(\langle A, \sigma \rangle, \langle B, \rho \rangle) \in R$ the conditions in Definition~\ref{def:strongsb} hold.
		\begin{enumerate}[i)]
			\item If $\langle A, \sigma \rangle \downarrow_c$, then we have $\langle B, \rho \rangle \downarrow_c$ by the hypothesis
			$\langle A, \sigma\rangle \sim \langle B, \rho\rangle$.
			\item Suppose that $\langle A, \sigma\rangle \rightarrow \langle A', \sigma' \rangle$. By Proposition~\ref{cor:true}
			we have $\langle A, \sigma\rangle \xrightarrow{\;  \bot \: } \langle A', \sigma' \rangle$.
			Since $\langle A, \sigma\rangle \sim \langle B, \rho\rangle$, then $\langle B, \rho \otimes \bot \rangle \rightarrow \langle B', \rho'\rangle$ with $\langle A', \sigma'\rangle \sim \langle B', \rho' \rangle$.
			Since $\rho = \rho \otimes \bot$, we have $\langle B, \rho  \rangle \rightarrow \langle B', \rho'\rangle$.
			% and $(\langle A', \sigma' \rangle, \langle B', \rho' \rangle) \in R$.
			\item By  Lemma~\ref{lemma:equality},
			$(\langle A, \sigma \otimes c'\rangle, \langle B, \rho \otimes c'\rangle) \in R$ for all $c' \in C^{\otimes}$.
		\end{enumerate}
		\item[From $\sim_{s}$ to $\sim$.]
		We show that $R = \{ (\langle A, \sigma \rangle, \langle B, \rho \rangle) \mid  \langle A, \sigma\rangle \sim_{s} \langle B, \rho\rangle\}$
		is a strong bisimulation, i.e., if $(\langle A, \sigma \rangle, \langle B, \rho \rangle) \in R$
		the conditions in Definition~\ref{def:strong} hold.
		\begin{enumerate}[i)]
			\item If $\langle A, \sigma \rangle \downarrow_c$, then we have $\langle B, \rho \rangle \downarrow_c$ by the hypothesis $\langle A, \sigma\rangle \sim_{s} \langle B, \rho\rangle$.
			\item Suppose that $\langle A, \sigma\rangle \xrightarrow{\;  \alpha \: } \langle A', \sigma' \rangle$.
			Then by Lemma~\ref{lemma:soundness} we have $\langle A, \sigma \otimes \alpha \rangle \rightarrow \langle A', \sigma'\rangle$.
			Since $\langle A, \sigma\rangle \sim_{s} \langle B, \rho\rangle$,
			then  $\langle A, \sigma \otimes \alpha \rangle \sim_{s} \langle B, \rho \otimes \alpha\rangle$ and
			thus $\langle B, \rho \otimes \alpha \rangle \rightarrow \langle B', \rho'\rangle$
			with $\langle A', \sigma'\rangle \sim_{s} \langle B', \rho'\rangle$.
		\end{enumerate}
	\end{description} 
\end{proof}


To prove the correspondence between weak bisimulations, we need a result 
similar to Lemma~\ref{lemma:equality}. The key is the preservation of weak barbs by
the addition of constraints to the store, which is trivial in the strong case.

\begin{lemma}\label{pres}
Let $\langle A, \sigma \rangle$, $\langle B, \rho \rangle$ be configurations and $c, d \in C^\otimes$.
%
If $\langle A, \sigma \rangle \approx \langle B, \rho \rangle$ and $\langle A, \sigma \otimes d \rangle\downarrow_c$, then $\langle B, \rho \otimes d\rangle\Downarrow_c$.
\end{lemma}

\begin{proof}
	If $\langle A, \sigma \otimes a \rangle \downarrow_c$, then $c \leq \sigma \otimes d$. Since $\langle A, \sigma \rangle \approx \langle B, \rho \rangle$ and $\langle A, \sigma \rangle \downarrow_\sigma$,
	then there exists $\langle B', \rho' \rangle$ such that $\langle B, \rho \rangle  \rightarrow^* \langle B', \rho'\rangle$ and $\sigma \leq \exists_\Gamma \rho'$ for $\Gamma = fv(\langle B', \rho'\rangle) \setminus fv(\langle B, \rho\rangle)$.
	Let us assume, without loss of generality, that $\Gamma \cap sv(d) = \emptyset$;
	since reductions are monotone (item 4 of Lemma~\ref{mono}),
	we  have $\langle B, \rho \otimes d \rangle  \rightarrow^* \langle B', \rho' \otimes d\rangle$. 
	%
	Finally, $c \leq \sigma \otimes d = \sigma \otimes \exists_\Gamma d \leq \exists_\Gamma \rho' \otimes \exists_\Gamma d \leq \exists_\Gamma (\rho' \otimes d)$, hence $\langle B, \rho \otimes d\rangle\Downarrow_c$.
\end{proof}

The result below uses Lemma~\ref{pres} and rephrases the proof of Lemma~\ref{lemma:equality}.

\begin{lemma}
\label{lemma:new3}
Let $\langle A, \sigma\rangle$, $\langle B, \rho\rangle$ be configurations and $c \in C^\otimes$
 Moreover, let us assume $\mathcal{C}$ to be invertible. 
If $\langle A, \sigma\rangle \approx \langle B, \rho\rangle$, then $\langle A, \sigma \otimes c \rangle \approx \langle B, \rho \otimes c\rangle$.
\end{lemma}
%\begin{proof}
%AGGIUNGERE SKETCH
%\end{proof}

Now the theorem below is proved by precisely mimicking the proof for Theorem~ and using Lemma~\ref{lemma:new3} instead of Lemma~\ref{lemma:equality}.

\begin{theorem}
\label{th:wbisimiffwsbbisim}
$\approx_{\mathit{s}} \; \subseteq \; \approx  \; \subseteq \;$. Moreover, let $\mathcal{C}$ be invertible.
Then $\approx_{\mathit{s}} \; = \; \approx$.
\end{theorem}
%\begin{proof}
%AGGIUNGERE SKETCH
%\end{proof}


%To give some intuition about the above definition,let us recall that in $\langle A, \sigma \rangle \longrightarrow \gamma$ the
%label $\alpha$ represents minimal information from the environment that needs to be added to the store $\sigma$ to evolve
%from $\langle A, \sigma \rangle$ into $\gamma'$. We do not require the transitions from $\langle B, \rho\rangle$ to match $\lapha$.
%Instead \emph{ii)} requires something weaker: If $\alpha$ is added to the store $\rho$, it should be possible to reduce into some $\gamma''$
%that it is in bisimulation with $\gamma'$. This condition is weaker because $\alpha$ may not be a minimal information allowing a transition
%from ?Q, e? into a ??? in the bisimulation, as shown in the previous example.


\subsection{Labelled versus saturated semantics.} The main appeal of saturated semantics resides in 
 being a congruence and, in fact, the minimal congruence contained in standard bisimulation~\cite{barbedMontanari}. The main drawback of this approach is that it
is in principle necessary to check the behaviour of a process under every context. The problem is somewhat mitigated for SCCP, since it suffices to close the store with respect to any possible compact element (item 3 of Definition~\ref{def:strongsb}).
%
At the same time, checking the feasibility of a reduction may require some computational effort, either for solving the combinatorial problem 
associated with calculating $\sigma \otimes d$, or for verifying if $c \leq \sigma$, as with agent $\askp{c}{A}$. 

This is the reason for searching labelled semantics and suitable notions of bisimilarity that may alleviate such a burden. 
%
The intuition is to consider labels which somehow represent  the ``minimal context allowing a process to reduce'', so that a bisimilarity-checking algorithm in principle needs  to verify this minimal context only, instead of every one. 
%
The idea has been exploited in the framework of crisp CCP~\cite{pippo}, and it 
is based on~\cite{Leifer:00:CONCUR,Bonchi:09:FOSSACS}.

%According to condition \emph{3} in Definition~\ref{def:strongsb}, an algorithm has to check  $(\langle A,\sigma \otimes d\rangle, \langle B,\rho \otimes d \rangle) \in R$ for all $d \in \mathcal{C}^\otimes$: given $\langle A, \sigma \otimes d \rangle \longrightarrow \gamma_1$, then it has to check the existence of $\gamma_2$ such that $\langle B, \rho \otimes d \rangle \longrightarrow \gamma_2$ and $(\gamma_1, \gamma_2) \in R$ (for all $d \in \mathcal{C}^\otimes$). Instead, in case of its labelled version, if $\gamma_1 \xrightarrow{\; \; \alpha\;  \;} \gamma_1'$ then we only need to  check the existence of $\gamma_2'$ such that $\langle B, \rho \otimes \alpha \rangle \longrightarrow \gamma_2'$ and $(\gamma_1', \gamma_2') \in R$. For instance, if $B$ is $\mathit{ask}(c) \rrarrow B'$, the check to be performed is $c \leq \rho \otimes \alpha$, which involves the solution of the combinatorial problem associated with $\rho \otimes \alpha$, and comes with some computational effort.
%
%Let us explain some of the advantages of $\wbisim$ over $\wsatbis$ with the following examples.

\begin{example}
\label{ex:barbvslabbis1}
Let us consider the agents $\askp{c}{\stopp}$ and ${\stopp}$. To prove that they are weakly bisimilar, it has to be proved that 
$\G\,\wbisim\,\G'$ for configurations $\G = \config{\askp{c}{\stopp}}{\bot}$ and $\G' = \config{\stopp}{\bot}$.
%
Consider the following relation
\[ \R = \{ (\config{\askp{c}{\stopp}}{\bot}, \config{\stopp}{\bot}), (\config{\stopp}{c}, \config{\stopp}{c})\} \]
It is quite easy to prove that it is a bisimulation, and in fact the smallest one identifying the two configurations. It suffices to note that 
by definition $c \odiv \bot = c$.

In order to prove that $\G\,\wsatbis\,\G'$, instead, we surely need to consider an infinite relation. Indeed, the smallest saturated bisimulation equating the 
two configuration is given by the relation below
\[
\mathcal{S} = \{ (\config{\askp{c}{\stopp}}{d}, \config{\stopp}{d}), (\config{\stopp}{e}, \config{\stopp}{e}) \ |\ d, e \in C^\otimes\, \&\, c \leq e\} 
\]
The relation above clearly is a saturated bisimulation, but any naive automatic check for that property might involve rather complex 
calculations.
%Note also that at the same cost  it can be checked that $\mathcal{S}$ is a bisimulation: as before, it boils down to the fact that 
%$c \leq d \otimes (c \odiv d)$.

%Since $\R$ is finite it is straightforward to prove that $\G\,\wbisim\,\G'$ given that 
%the symmetric closure of $\R$ is a weak bisimulation as in Definition \ref{def:strong}.
%On the other hand, proving that $\G\,\wsatbis\,\G'$ is not so difficult (notice that $\mathcal{S}$ is very similar to $\R$),
%however one is obliged to consider an infinite relation even for comparing rather simple configurations such as $\G$ and $\G'$.
%\qed
\end{example}

Another reason for the complexity of checking saturated bisimilarity is the need of considering 
the closure $\longrightarrow^*$ of the reduction relation, which may cause a combinatorial explosion.
Think e.g. of the agents $\prod_{i \in I} \askp{c_i}{\stopp}$ and $\stopp$. Of course, they might be proved equivalent 
by exploiting the fact that saturated bisimilarity is a congruence, and by verifying that $\stopp \parallel A \,\wsatbis\, A $ for all 
the agents $A$.
%
A direct proof would instead require a check for each store of the reductions arising from all the possible interleaving of the $c_i$
elements.


%\begin{example}
%\label{ex:barbvslabbis2}
%Let $I$ be a set of indexes from $1$ to $n$, i.e. $I = \{1,\dots,n\}$. Now let $c_i \in C^\otimes$ such that
%for all $i, j$ if $i \neq j$ then $c_i \not\leq c_j$. Now consider the following configurations $\G$ and $\G'$:
%\[
%\G = \config{\prod_{i \in I} \askp{c_i}{\stopp}}{\bot} \ \ \ \ \G' = \config{\stopp}{\bot}
%\]
%Let us show first that $\G\,\wbisim\,\G'$ for this purpose consider the following relation:
%\[
%\R = \{ (\config{\prod_{j \in J} \askp{c_j}{\stopp}}{\bigotimes_{k \in K} c_k}, \config{\stopp}{\bigotimes_{k \in K} c_k}) 
%\ |\  J \subseteq 2^I \mbox{ and } K = J - I\}
%\]
%To prove that $\R$ is a weak bisimulation we need to check the two conditions in Definition \ref{def:weak}). Condition 1 is trivial
%since all the pair have the same store. For condition 2 we must consider the following transition:
%\[
% \config{\prod_{j \in J} \askp{c_j}{\stopp}}{\bigotimes_{k \in K} c_k} \trans{c_i} 
% \config{\prod_{j \in J-\{i\}} \askp{c_j}{\stopp}}{\bigotimes_{k \in K+\{i\}} c_k} \mbox{ for some } i \in J
%\]
%And notice that after the transition the resulting configuration is related with $\G'$ in $\R$ by definition. 
%Hence $\R$ is a weak bisimulation and since $(\G, \G') \in \R$ (when $J = I$) then $\G\,\wbisim\,\G'$.
%
%In turn, let us prove that $\G\,\wsatbis\,\G'$, in order to do this consider the following relation:
%\[
%\mathcal{S} = \{ (\config{\prod_{j \in J} \askp{c_j}{\stopp}}{d}, \config{\stopp}{d}) \ |\  J \subseteq 2^I\}
%\]
%Note that $\mathcal{S}$ is a more succint relation, however before jumping to conclusions
%let us prove that it is a weak saturated barbed bisimulation as in Definition \ref{def:weaksb}.
%If we take the configuration $\G'' = \config{\prod_{j \in J} \askp{c_j}{\stopp}}{d}$ we have to consider about $2^{|J|}$
%cases in order to consider every possible interleaving that $\G''$,
%namely $d$ must take each value in the set $\bigotimes_{k \in K} c_k$ %
%where $K \subseteq 2^{|J|}$. Recall that $\wsatbis$ uses the reflexive and transitive closure $\reds$.
%Hence it is clearly much more difficult to prove that $\mathcal{S}$ is a weak saturated %
%barbed bisimulation which instead was proven easily by using the labeled semantics.
%\qed
%\end{example

\end{document}
