\documentclass[main.tex]{subfiles}
\begin{document}

\section{Conclusions and Further Work}\label{sec:conclusions}
Inspired by the characterisation of the labelled semantics for the crisp variant of the language~\cite{pippo}, 
in this paper we investigated observational and behavioural 
equivalences of the deterministic fragment of soft CCP~\cite{scc}: we introduced 
the notion of $\otimes$-compactness, we proposed an observational semantics for the language  and 
presented a novel behavioural characterisation in terms of weak bisimilarity 
(enhancing the syntactic bisimulation advanced in \cite{fun14}).
 We then rounded the work by introducing 
 %following the seminal work of Saraswat, Rinard, and Panagaden: 
 % we recast the notion of compactness with respect to $\otimes$, and we provide 
 an axiomatisation of the finite fragment of the language, in the spirit of \cite{popl91}. 
 %Axiomatisation for SCCP is novel, while only syntactic bisimulation has been advanced in \cite{fun14}.
 
The use of residuation theory (advanced in~\cite{ecai06} for solving soft constraints problems)
provides an elegant way to define the minimal information 
that enables the firing of actions in the LTS shown in Sec.~\ref{sec:4}. 
 %
 This choice allowed for the study of the behavioural equivalence of agents in terms of weak and strong bisimilarity on such LTS, 
 and it allowed for relating them to the corresponding barbed bisimilarities of (unlabelled) reductions
 and with the standard semantics via fair computations.
 %
 The two kinds of equivalences, as well as the axiomatisation for weak bisimilarity, are presented in this paper for the first time.

For future work, we plan to provide a denotational semantics for soft CCP by building on the work for the crisp case in \cite{popl91}. 
%
No denotational semantics has been yet proposed for \cite{scc}.
Concerning the axioms, we will try and investigate the relationship between soft CCP and a logical system whose fundamental 
properties are closely related to the ones we have investigated
in this paper; namely \emph{affine linear logic} \cite{LagoM04}. %Asperti98,
%
This logical system  \emph{rejects contraction} 
%(or idempotence of entailment)
but \emph{admits weakening}, which intuitively correspond to dropping idempotence and preserving
monotonicity in the soft formalism.  %Our quest for a complete axiomatisation of soft CCP may benefit from the proof theory of affine linear logic.  
The denotational model of CCP is based on \emph{closure operators}: Each 
agent is compositionally interpreted as a monotonic, extensive and idempotent operator/function on constraints.
%
We shall then investigate a denotational 
model for soft CCP  processes based on \emph{pre-closure operators}~\cite{general-topology}.
% (or \emph{\v{C}ech closure operators}), which are not required to be idempotent. 

We plan to consider two extensions of the language, checking how far the results given 
in this paper can be adapted. As evidenced by \cite{Pino:14:ICTAC},
%,PinoPhD}, 
a non-deterministic extension is an 
interesting challenge since the closure under any context for the saturated 
bisimilarity gets more elaborated than just closing
with respect to the addition of constraints (Defs. \ref{def:strongsb} and \ref{def:weaksb}, condition 3), and similarly
one also needs to find the right formulation of bisimilarity for the labelled transitions systems. 
%
Also, the presence of residuation makes intuitive the definition of a retract operator for the calculus. 
%
Even if the operational semantics would be less affected, retraction would require a complete reformulation of the semantics via fair computations, since monotonicity
(as stated in Lemma~\ref{mono}) would not hold~\cite{sefm12,fun14}.
%%%%%%%%%%%%%%%%%%%%%%%%%%%
%Moreover, we might consider languages with temporal features, such as \emph{timed} SCCP~\cite{coordination08}, where a
%reduction takes a bounded period of time and it is measured by a discrete global clock. Maximal parallel steps are adopted there with a 
%new construct that can e.g. express time-out and pre-emption, and developing suitable 
%temporal variants of bisimilarity might reveal a worthwhile, albeit difficult task.

%\textcolor{red}{
Finally, we would like to find the same observational equivalences also for a semantics that considers local stores of agents. In this case, the hiding operator needs to carry some information on the variables it abstracts.
According to~\cite{extendedHiding}, we should consider an extended operator $\exists_x^\sigma$, for $\sigma$ the local store. The intuition is that variable $x$ may be local to a component $\exists_x \sigma'$ of the store $\sigma$, yet visible at a global level: we must then evaluate $A$ in the store
when the local $x$ is hidden, yet the possible duplications are removed (e.g., $\exists_x \sigma'$ may already occur in the global store $\sigma$). The final store contains in $\sigma_1$ the original $\sigma'$ increased by what has been added by the step.
%
To this end, we consider also to investigate the more general framework for distilling labelled semantics from unlabelled ones proposed 
in~\cite{BonchiGM14}.

%}
%%%%%%%%%%%%%%%%%%%%%%%%%%%


%Note that in classical SCCP, \emph{ask} and \emph{tell} operations are equipped with a monoid element $a$ or a constraint 
%$\phi$ that are used as a cut level to prune computations that are not good enough. These checks lead to a \emph{fail} state 
%whenever they are not satisfied by the current store. Our intent is to leave this study (not treated in this paper) as a future work.
%
%Moreover, in the future we also plan to consider a non-deterministic extension of the language, and thus to study how the bisimilarity 
%equivalences we give in this paper adapt in it. Indeed we can also develop the denotational semantics, currently missing in SCCP 
%language, by trying to follow the same approach given for CCP~\cite{popl91}.
\end{document}
